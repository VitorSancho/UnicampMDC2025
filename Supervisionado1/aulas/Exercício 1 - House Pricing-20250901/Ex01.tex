% Options for packages loaded elsewhere
\PassOptionsToPackage{unicode}{hyperref}
\PassOptionsToPackage{hyphens}{url}
%
\documentclass[
]{article}
\usepackage{amsmath,amssymb}
\usepackage{iftex}
\ifPDFTeX
  \usepackage[T1]{fontenc}
  \usepackage[utf8]{inputenc}
  \usepackage{textcomp} % provide euro and other symbols
\else % if luatex or xetex
  \usepackage{unicode-math} % this also loads fontspec
  \defaultfontfeatures{Scale=MatchLowercase}
  \defaultfontfeatures[\rmfamily]{Ligatures=TeX,Scale=1}
\fi
\usepackage{lmodern}
\ifPDFTeX\else
  % xetex/luatex font selection
\fi
% Use upquote if available, for straight quotes in verbatim environments
\IfFileExists{upquote.sty}{\usepackage{upquote}}{}
\IfFileExists{microtype.sty}{% use microtype if available
  \usepackage[]{microtype}
  \UseMicrotypeSet[protrusion]{basicmath} % disable protrusion for tt fonts
}{}
\makeatletter
\@ifundefined{KOMAClassName}{% if non-KOMA class
  \IfFileExists{parskip.sty}{%
    \usepackage{parskip}
  }{% else
    \setlength{\parindent}{0pt}
    \setlength{\parskip}{6pt plus 2pt minus 1pt}}
}{% if KOMA class
  \KOMAoptions{parskip=half}}
\makeatother
\usepackage{xcolor}
\usepackage[margin=1in]{geometry}
\usepackage{color}
\usepackage{fancyvrb}
\newcommand{\VerbBar}{|}
\newcommand{\VERB}{\Verb[commandchars=\\\{\}]}
\DefineVerbatimEnvironment{Highlighting}{Verbatim}{commandchars=\\\{\}}
% Add ',fontsize=\small' for more characters per line
\usepackage{framed}
\definecolor{shadecolor}{RGB}{248,248,248}
\newenvironment{Shaded}{\begin{snugshade}}{\end{snugshade}}
\newcommand{\AlertTok}[1]{\textcolor[rgb]{0.94,0.16,0.16}{#1}}
\newcommand{\AnnotationTok}[1]{\textcolor[rgb]{0.56,0.35,0.01}{\textbf{\textit{#1}}}}
\newcommand{\AttributeTok}[1]{\textcolor[rgb]{0.13,0.29,0.53}{#1}}
\newcommand{\BaseNTok}[1]{\textcolor[rgb]{0.00,0.00,0.81}{#1}}
\newcommand{\BuiltInTok}[1]{#1}
\newcommand{\CharTok}[1]{\textcolor[rgb]{0.31,0.60,0.02}{#1}}
\newcommand{\CommentTok}[1]{\textcolor[rgb]{0.56,0.35,0.01}{\textit{#1}}}
\newcommand{\CommentVarTok}[1]{\textcolor[rgb]{0.56,0.35,0.01}{\textbf{\textit{#1}}}}
\newcommand{\ConstantTok}[1]{\textcolor[rgb]{0.56,0.35,0.01}{#1}}
\newcommand{\ControlFlowTok}[1]{\textcolor[rgb]{0.13,0.29,0.53}{\textbf{#1}}}
\newcommand{\DataTypeTok}[1]{\textcolor[rgb]{0.13,0.29,0.53}{#1}}
\newcommand{\DecValTok}[1]{\textcolor[rgb]{0.00,0.00,0.81}{#1}}
\newcommand{\DocumentationTok}[1]{\textcolor[rgb]{0.56,0.35,0.01}{\textbf{\textit{#1}}}}
\newcommand{\ErrorTok}[1]{\textcolor[rgb]{0.64,0.00,0.00}{\textbf{#1}}}
\newcommand{\ExtensionTok}[1]{#1}
\newcommand{\FloatTok}[1]{\textcolor[rgb]{0.00,0.00,0.81}{#1}}
\newcommand{\FunctionTok}[1]{\textcolor[rgb]{0.13,0.29,0.53}{\textbf{#1}}}
\newcommand{\ImportTok}[1]{#1}
\newcommand{\InformationTok}[1]{\textcolor[rgb]{0.56,0.35,0.01}{\textbf{\textit{#1}}}}
\newcommand{\KeywordTok}[1]{\textcolor[rgb]{0.13,0.29,0.53}{\textbf{#1}}}
\newcommand{\NormalTok}[1]{#1}
\newcommand{\OperatorTok}[1]{\textcolor[rgb]{0.81,0.36,0.00}{\textbf{#1}}}
\newcommand{\OtherTok}[1]{\textcolor[rgb]{0.56,0.35,0.01}{#1}}
\newcommand{\PreprocessorTok}[1]{\textcolor[rgb]{0.56,0.35,0.01}{\textit{#1}}}
\newcommand{\RegionMarkerTok}[1]{#1}
\newcommand{\SpecialCharTok}[1]{\textcolor[rgb]{0.81,0.36,0.00}{\textbf{#1}}}
\newcommand{\SpecialStringTok}[1]{\textcolor[rgb]{0.31,0.60,0.02}{#1}}
\newcommand{\StringTok}[1]{\textcolor[rgb]{0.31,0.60,0.02}{#1}}
\newcommand{\VariableTok}[1]{\textcolor[rgb]{0.00,0.00,0.00}{#1}}
\newcommand{\VerbatimStringTok}[1]{\textcolor[rgb]{0.31,0.60,0.02}{#1}}
\newcommand{\WarningTok}[1]{\textcolor[rgb]{0.56,0.35,0.01}{\textbf{\textit{#1}}}}
\usepackage{graphicx}
\makeatletter
\newsavebox\pandoc@box
\newcommand*\pandocbounded[1]{% scales image to fit in text height/width
  \sbox\pandoc@box{#1}%
  \Gscale@div\@tempa{\textheight}{\dimexpr\ht\pandoc@box+\dp\pandoc@box\relax}%
  \Gscale@div\@tempb{\linewidth}{\wd\pandoc@box}%
  \ifdim\@tempb\p@<\@tempa\p@\let\@tempa\@tempb\fi% select the smaller of both
  \ifdim\@tempa\p@<\p@\scalebox{\@tempa}{\usebox\pandoc@box}%
  \else\usebox{\pandoc@box}%
  \fi%
}
% Set default figure placement to htbp
\def\fps@figure{htbp}
\makeatother
\usepackage{svg}
\setlength{\emergencystretch}{3em} % prevent overfull lines
\providecommand{\tightlist}{%
  \setlength{\itemsep}{0pt}\setlength{\parskip}{0pt}}
\setcounter{secnumdepth}{-\maxdimen} % remove section numbering
\usepackage{bookmark}
\IfFileExists{xurl.sty}{\usepackage{xurl}}{} % add URL line breaks if available
\urlstyle{same}
\hypersetup{
  pdftitle={INF0615 -- Aprendizado Supervisionado I},
  hidelinks,
  pdfcreator={LaTeX via pandoc}}

\title{INF0615 -- Aprendizado Supervisionado I}
\usepackage{etoolbox}
\makeatletter
\providecommand{\subtitle}[1]{% add subtitle to \maketitle
  \apptocmd{\@title}{\par {\large #1 \par}}{}{}
}
\makeatother
\subtitle{Exercício 01 - House Pricing}
\author{}
\date{\vspace{-2.5em}}

\begin{document}
\maketitle

Neste exercício, iremos explorar conceitos fundamentais de aprendizado
supervisionado, aplicando-os a um problema de \textbf{precificação de
imóveis}. Trabalharemos com as seguintes técnicas:

\begin{itemize}
\tightlist
\item
  \textbf{Regressão Linear}: Modelagem da relação entre variáveis
  preditoras e o preço do imóvel.
\item
  \textbf{Combinação de Features}: Uso de transformações e combinações
  de atributos para melhorar a performance do modelo.
\item
  \textbf{Regressão Polinomial}: Introdução de termos não lineares para
  capturar padrões mais complexos nos dados.
\end{itemize}

\begin{center}\rule{0.5\linewidth}{0.5pt}\end{center}

\subsection{Módulo 0: Instalando
Dependências}\label{muxf3dulo-0-instalando-dependuxeancias}

Antes de prosseguir, verifique se as bibliotecas necessárias estão
instaladas. Caso não estejam, remova o \texttt{\#} das linhas abaixo
para instalá-las:

\begin{Shaded}
\begin{Highlighting}[]
\FunctionTok{install.packages}\NormalTok{(}\StringTok{"ggplot2"}\NormalTok{)  }\CommentTok{\# Para visualização de dados}
\FunctionTok{install.packages}\NormalTok{(}\StringTok{"reshape2"}\NormalTok{) }\CommentTok{\# Para manipulação de dados}
\FunctionTok{install.packages}\NormalTok{(}\StringTok{"corrplot"}\NormalTok{) }\CommentTok{\# Para visualização de correlações}
\end{Highlighting}
\end{Shaded}

\begin{center}\rule{0.5\linewidth}{0.5pt}\end{center}

\subsection{Módulo 1: Configuração do
Ambiente}\label{muxf3dulo-1-configurauxe7uxe3o-do-ambiente}

Carrega as bibliotecas necessárias e define uma semente aleatória para
reprodutibilidade:

\begin{Shaded}
\begin{Highlighting}[]
\FunctionTok{library}\NormalTok{(ggplot2)  }\CommentTok{\# Visualização de dados}
\FunctionTok{library}\NormalTok{(reshape2) }\CommentTok{\# Manipulação de dados}
\FunctionTok{library}\NormalTok{(corrplot) }\CommentTok{\# Visualização de correlações}
\FunctionTok{set.seed}\NormalTok{(}\DecValTok{42}\NormalTok{)      }\CommentTok{\# Garantir reprodutibilidade dos experimentos}
\end{Highlighting}
\end{Shaded}

\begin{center}\rule{0.5\linewidth}{0.5pt}\end{center}

\subsection{Módulo 2: Carregamento do
Dataset}\label{muxf3dulo-2-carregamento-do-dataset}

Carregue os conjuntos de dados de \textbf{treinamento} e
\textbf{validação}, que serão utilizados para treinar e avaliar os
modelos:

\begin{Shaded}
\begin{Highlighting}[]
\NormalTok{train\_data }\OtherTok{\textless{}{-}} \FunctionTok{read.csv}\NormalTok{(}\StringTok{"housePricing\_train\_set.csv"}\NormalTok{, }
                       \AttributeTok{header=}\ConstantTok{TRUE}\NormalTok{, }\AttributeTok{stringsAsFactors=}\ConstantTok{TRUE}\NormalTok{)}

\NormalTok{valid\_data }\OtherTok{\textless{}{-}} \FunctionTok{read.csv}\NormalTok{(}\StringTok{"housePricing\_val\_set.csv"}\NormalTok{, }
                       \AttributeTok{header=}\ConstantTok{TRUE}\NormalTok{, }\AttributeTok{stringsAsFactors=}\ConstantTok{TRUE}\NormalTok{)}
\end{Highlighting}
\end{Shaded}

\subsubsection{Módulo 2.1: Explorando os
Dados}\label{muxf3dulo-2.1-explorando-os-dados}

A seguir, vamos analisar algumas informações básicas sobre os conjuntos
de dados.

\paragraph{Módulo 2.1.1: Dados de
Treinamento}\label{muxf3dulo-2.1.1-dados-de-treinamento}

\begin{Shaded}
\begin{Highlighting}[]
\CommentTok{\# Exibir as primeiras linhas do dataset}
\NormalTok{display\_head }\OtherTok{\textless{}{-}} \FunctionTok{head}\NormalTok{(train\_data)}
\FunctionTok{print}\NormalTok{(display\_head)}
\end{Highlighting}
\end{Shaded}

\begin{verbatim}
##   longitude latitude housing_median_age total_rooms total_bedrooms population
## 1      -118     34.0                 52        1754            452       1849
## 2      -118     34.0                 46        1217            322        662
## 3      -122     38.0                 33          44              6         23
## 4      -119     35.4                 35         120             35        477
## 5      -119     34.2                 33        5252            760       2041
## 6      -118     33.5                  4        3623            734       1129
##   households median_income median_house_value ocean_proximity
## 1        445          2.37             122800       <1H OCEAN
## 2        305          3.17             140300       <1H OCEAN
## 3         11          4.12             212500        NEAR BAY
## 4         41          1.91              47500          INLAND
## 5        730          6.80             389700       <1H OCEAN
## 6        530          5.73             500001       <1H OCEAN
\end{verbatim}

\begin{Shaded}
\begin{Highlighting}[]
\CommentTok{\# Dimensões do dataset}
\FunctionTok{cat}\NormalTok{(}\StringTok{"Amostras:"}\NormalTok{, }\FunctionTok{nrow}\NormalTok{(train\_data), }\StringTok{"}\SpecialCharTok{\textbackslash{}n}\StringTok{Features:"}\NormalTok{, }\FunctionTok{ncol}\NormalTok{(train\_data)}\SpecialCharTok{{-}}\DecValTok{1}\NormalTok{, }\StringTok{"}\SpecialCharTok{\textbackslash{}n}\StringTok{"}\NormalTok{)}
\end{Highlighting}
\end{Shaded}

\begin{verbatim}
## Amostras: 12268 
## Features: 9
\end{verbatim}

\begin{Shaded}
\begin{Highlighting}[]
\CommentTok{\# Resumo estatístico dos dados}
\FunctionTok{summary}\NormalTok{(train\_data)}
\end{Highlighting}
\end{Shaded}

\begin{verbatim}
##    longitude       latitude    housing_median_age  total_rooms   
##  Min.   :-124   Min.   :32.5   Min.   : 1.0       Min.   :   15  
##  1st Qu.:-122   1st Qu.:33.9   1st Qu.:18.0       1st Qu.: 1461  
##  Median :-119   Median :34.3   Median :29.0       Median : 2130  
##  Mean   :-120   Mean   :35.6   Mean   :28.8       Mean   : 2641  
##  3rd Qu.:-118   3rd Qu.:37.7   3rd Qu.:37.0       3rd Qu.: 3131  
##  Max.   :-114   Max.   :42.0   Max.   :52.0       Max.   :37937  
##  total_bedrooms   population      households   median_income  
##  Min.   :   3   Min.   :    3   Min.   :   2   Min.   : 0.50  
##  1st Qu.: 297   1st Qu.:  793   1st Qu.: 281   1st Qu.: 2.56  
##  Median : 437   Median : 1169   Median : 411   Median : 3.54  
##  Mean   : 539   Mean   : 1430   Mean   : 500   Mean   : 3.87  
##  3rd Qu.: 647   3rd Qu.: 1720   3rd Qu.: 603   3rd Qu.: 4.76  
##  Max.   :6445   Max.   :35682   Max.   :6082   Max.   :15.00  
##  median_house_value   ocean_proximity
##  Min.   : 14999     <1H OCEAN :5361  
##  1st Qu.:119600     INLAND    :3892  
##  Median :180300     ISLAND    :   3  
##  Mean   :206918     NEAR BAY  :1396  
##  3rd Qu.:265000     NEAR OCEAN:1616  
##  Max.   :500001
\end{verbatim}

\begin{Shaded}
\begin{Highlighting}[]
\CommentTok{\# Verificar se há valores ausentes}
\ControlFlowTok{if}\NormalTok{ (}\FunctionTok{any}\NormalTok{(}\FunctionTok{is.na}\NormalTok{(train\_data))) \{}
  \FunctionTok{cat}\NormalTok{(}\StringTok{"Aviso: Existem valores ausentes no conjunto de treinamento.}\SpecialCharTok{\textbackslash{}n}\StringTok{"}\NormalTok{)}
\NormalTok{\} }\ControlFlowTok{else}\NormalTok{ \{}
  \FunctionTok{cat}\NormalTok{(}\StringTok{"Nenhum valor ausente encontrado no conjunto de treinamento.}\SpecialCharTok{\textbackslash{}n}\StringTok{"}\NormalTok{)}
\NormalTok{\}}
\end{Highlighting}
\end{Shaded}

\begin{verbatim}
## Nenhum valor ausente encontrado no conjunto de treinamento.
\end{verbatim}

\paragraph{Módulo 2.1.2: Dados de
Validação}\label{muxf3dulo-2.1.2-dados-de-validauxe7uxe3o}

\begin{Shaded}
\begin{Highlighting}[]
\CommentTok{\# Exibir as primeiras linhas do dataset}
\NormalTok{display\_head }\OtherTok{\textless{}{-}} \FunctionTok{head}\NormalTok{(valid\_data)}
\FunctionTok{print}\NormalTok{(display\_head)}
\end{Highlighting}
\end{Shaded}

\begin{verbatim}
##   longitude latitude housing_median_age total_rooms total_bedrooms population
## 1      -116     33.7                 14        2774            566       1530
## 2      -122     37.2                 12       10236           1878       5674
## 3      -119     34.1                 35        3795            690       1521
## 4      -118     34.2                 19        9259           1653       3963
## 5      -117     33.0                  5        2276            311       1158
## 6      -118     34.1                 35        2501            651       1182
##   households median_income median_house_value ocean_proximity
## 1        505          3.07             104100          INLAND
## 2       1816          4.75             261100       <1H OCEAN
## 3        653          5.87             448100       <1H OCEAN
## 4       1595          6.00             228700       <1H OCEAN
## 5        317          6.43             271900       <1H OCEAN
## 6        591          1.45             113200          INLAND
\end{verbatim}

\begin{Shaded}
\begin{Highlighting}[]
\CommentTok{\# Dimensões do dataset}
\FunctionTok{cat}\NormalTok{(}\StringTok{"Amostras:"}\NormalTok{, }\FunctionTok{nrow}\NormalTok{(valid\_data), }\StringTok{"}\SpecialCharTok{\textbackslash{}n}\StringTok{Features:"}\NormalTok{, }\FunctionTok{ncol}\NormalTok{(valid\_data)}\SpecialCharTok{{-}}\DecValTok{1}\NormalTok{, }\StringTok{"}\SpecialCharTok{\textbackslash{}n}\StringTok{"}\NormalTok{)}
\end{Highlighting}
\end{Shaded}

\begin{verbatim}
## Amostras: 4084 
## Features: 9
\end{verbatim}

\begin{Shaded}
\begin{Highlighting}[]
\CommentTok{\# Resumo estatístico dos dados}
\FunctionTok{summary}\NormalTok{(valid\_data)}
\end{Highlighting}
\end{Shaded}

\begin{verbatim}
##    longitude       latitude    housing_median_age  total_rooms   
##  Min.   :-124   Min.   :32.5   Min.   : 1.0       Min.   :    6  
##  1st Qu.:-122   1st Qu.:33.9   1st Qu.:18.0       1st Qu.: 1465  
##  Median :-118   Median :34.3   Median :28.0       Median : 2152  
##  Mean   :-120   Mean   :35.6   Mean   :28.2       Mean   : 2648  
##  3rd Qu.:-118   3rd Qu.:37.7   3rd Qu.:36.0       3rd Qu.: 3183  
##  Max.   :-115   Max.   :42.0   Max.   :52.0       Max.   :39320  
##  total_bedrooms   population      households   median_income  
##  Min.   :   2   Min.   :    8   Min.   :   2   Min.   : 0.50  
##  1st Qu.: 300   1st Qu.:  810   1st Qu.: 285   1st Qu.: 2.56  
##  Median : 438   Median : 1168   Median : 416   Median : 3.51  
##  Mean   : 542   Mean   : 1425   Mean   : 504   Mean   : 3.87  
##  3rd Qu.: 658   3rd Qu.: 1743   3rd Qu.: 614   3rd Qu.: 4.72  
##  Max.   :6210   Max.   :16305   Max.   :5358   Max.   :15.00  
##  median_house_value   ocean_proximity
##  Min.   : 14999     <1H OCEAN :1827  
##  1st Qu.:120000     INLAND    :1325  
##  Median :180950     NEAR BAY  : 423  
##  Mean   :207295     NEAR OCEAN: 509  
##  3rd Qu.:265200                      
##  Max.   :500001
\end{verbatim}

\begin{Shaded}
\begin{Highlighting}[]
\CommentTok{\# Verificar se há valores ausentes}
\ControlFlowTok{if}\NormalTok{ (}\FunctionTok{any}\NormalTok{(}\FunctionTok{is.na}\NormalTok{(valid\_data))) \{}
  \FunctionTok{cat}\NormalTok{(}\StringTok{"Aviso: Existem valores ausentes no conjunto de validação.}\SpecialCharTok{\textbackslash{}n}\StringTok{"}\NormalTok{)}
\NormalTok{\} }\ControlFlowTok{else}\NormalTok{ \{}
  \FunctionTok{cat}\NormalTok{(}\StringTok{"Nenhum valor ausente encontrado no conjunto de validação.}\SpecialCharTok{\textbackslash{}n}\StringTok{"}\NormalTok{)}
\NormalTok{\}}
\end{Highlighting}
\end{Shaded}

\begin{verbatim}
## Nenhum valor ausente encontrado no conjunto de validação.
\end{verbatim}

\subsubsection{Módulo 2.2:
One-Hot-Encoding}\label{muxf3dulo-2.2-one-hot-encoding}

One-Hot-Encoding é um método de transformação de variáveis categóricas
em variáveis numéricas, no qual cada categoria é representada como uma
coluna binária (0 ou 1).

\begin{figure}
\centering
\pandocbounded{\includesvg[keepaspectratio]{OneHotEncoder.svg}}
\caption{Fig 1. Exemplo de One Hot Encoder}
\end{figure}

Aqui, temos dois exemplos de como realizar essa transformação:
manualmente e automaticamente.

\paragraph{Módulo 2.2.1: Transformação
Manual}\label{muxf3dulo-2.2.1-transformauxe7uxe3o-manual}

\begin{Shaded}
\begin{Highlighting}[]
\NormalTok{train\_encoded }\OtherTok{\textless{}{-}}\NormalTok{ train\_data}
\NormalTok{valid\_encoded }\OtherTok{\textless{}{-}}\NormalTok{ valid\_data}

\CommentTok{\# Criando colunas binárias para cada categoria de "ocean\_proximity"}
\NormalTok{categories     }\OtherTok{\textless{}{-}} \FunctionTok{c}\NormalTok{(}\StringTok{"\textless{}1H OCEAN"}\NormalTok{, }\StringTok{"INLAND"}\NormalTok{, }\StringTok{"ISLAND"}\NormalTok{, }\StringTok{"NEAR BAY"}\NormalTok{, }\StringTok{"NEAR OCEAN"}\NormalTok{)}
\NormalTok{new\_categories }\OtherTok{\textless{}{-}} \FunctionTok{c}\NormalTok{(}\StringTok{\textquotesingle{}less1Hocean\textquotesingle{}}\NormalTok{, }\StringTok{\textquotesingle{}inland\textquotesingle{}}\NormalTok{, }\StringTok{\textquotesingle{}island\textquotesingle{}}\NormalTok{, }\StringTok{\textquotesingle{}nearbay\textquotesingle{}}\NormalTok{, }\StringTok{\textquotesingle{}nearocean\textquotesingle{}}\NormalTok{)}

\CommentTok{\# Looping sobre as categorias}
\ControlFlowTok{for}\NormalTok{ (i }\ControlFlowTok{in} \FunctionTok{seq\_along}\NormalTok{(categories)) \{}
\NormalTok{  cat     }\OtherTok{\textless{}{-}}\NormalTok{ categories[i]}
\NormalTok{  new\_cat }\OtherTok{\textless{}{-}}\NormalTok{ new\_categories[i]}
  
  \CommentTok{\# Criando colunas binárias para o treino e a validação}
\NormalTok{  train\_encoded[[new\_cat]] }\OtherTok{\textless{}{-}} \FunctionTok{as.numeric}\NormalTok{(train\_encoded}\SpecialCharTok{$}\NormalTok{ocean\_proximity }\SpecialCharTok{==}\NormalTok{ cat)}
\NormalTok{  valid\_encoded[[new\_cat]] }\OtherTok{\textless{}{-}} \FunctionTok{as.numeric}\NormalTok{(valid\_encoded}\SpecialCharTok{$}\NormalTok{ocean\_proximity }\SpecialCharTok{==}\NormalTok{ cat)}
\NormalTok{\}}

\CommentTok{\# Removendo a coluna original categórica}
\NormalTok{train\_encoded}\SpecialCharTok{$}\NormalTok{ocean\_proximity }\OtherTok{\textless{}{-}} \ConstantTok{NULL}
\NormalTok{valid\_encoded}\SpecialCharTok{$}\NormalTok{ocean\_proximity }\OtherTok{\textless{}{-}} \ConstantTok{NULL}
\end{Highlighting}
\end{Shaded}

\begin{Shaded}
\begin{Highlighting}[]
\FunctionTok{summary}\NormalTok{(train\_encoded)}
\end{Highlighting}
\end{Shaded}

\begin{verbatim}
##    longitude       latitude    housing_median_age  total_rooms   
##  Min.   :-124   Min.   :32.5   Min.   : 1.0       Min.   :   15  
##  1st Qu.:-122   1st Qu.:33.9   1st Qu.:18.0       1st Qu.: 1461  
##  Median :-119   Median :34.3   Median :29.0       Median : 2130  
##  Mean   :-120   Mean   :35.6   Mean   :28.8       Mean   : 2641  
##  3rd Qu.:-118   3rd Qu.:37.7   3rd Qu.:37.0       3rd Qu.: 3131  
##  Max.   :-114   Max.   :42.0   Max.   :52.0       Max.   :37937  
##  total_bedrooms   population      households   median_income  
##  Min.   :   3   Min.   :    3   Min.   :   2   Min.   : 0.50  
##  1st Qu.: 297   1st Qu.:  793   1st Qu.: 281   1st Qu.: 2.56  
##  Median : 437   Median : 1169   Median : 411   Median : 3.54  
##  Mean   : 539   Mean   : 1430   Mean   : 500   Mean   : 3.87  
##  3rd Qu.: 647   3rd Qu.: 1720   3rd Qu.: 603   3rd Qu.: 4.76  
##  Max.   :6445   Max.   :35682   Max.   :6082   Max.   :15.00  
##  median_house_value  less1Hocean        inland          island        
##  Min.   : 14999     Min.   :0.000   Min.   :0.000   Min.   :0.000000  
##  1st Qu.:119600     1st Qu.:0.000   1st Qu.:0.000   1st Qu.:0.000000  
##  Median :180300     Median :0.000   Median :0.000   Median :0.000000  
##  Mean   :206918     Mean   :0.437   Mean   :0.317   Mean   :0.000244  
##  3rd Qu.:265000     3rd Qu.:1.000   3rd Qu.:1.000   3rd Qu.:0.000000  
##  Max.   :500001     Max.   :1.000   Max.   :1.000   Max.   :1.000000  
##     nearbay        nearocean    
##  Min.   :0.000   Min.   :0.000  
##  1st Qu.:0.000   1st Qu.:0.000  
##  Median :0.000   Median :0.000  
##  Mean   :0.114   Mean   :0.132  
##  3rd Qu.:0.000   3rd Qu.:0.000  
##  Max.   :1.000   Max.   :1.000
\end{verbatim}

\begin{Shaded}
\begin{Highlighting}[]
\FunctionTok{summary}\NormalTok{(valid\_encoded)}
\end{Highlighting}
\end{Shaded}

\begin{verbatim}
##    longitude       latitude    housing_median_age  total_rooms   
##  Min.   :-124   Min.   :32.5   Min.   : 1.0       Min.   :    6  
##  1st Qu.:-122   1st Qu.:33.9   1st Qu.:18.0       1st Qu.: 1465  
##  Median :-118   Median :34.3   Median :28.0       Median : 2152  
##  Mean   :-120   Mean   :35.6   Mean   :28.2       Mean   : 2648  
##  3rd Qu.:-118   3rd Qu.:37.7   3rd Qu.:36.0       3rd Qu.: 3183  
##  Max.   :-115   Max.   :42.0   Max.   :52.0       Max.   :39320  
##  total_bedrooms   population      households   median_income  
##  Min.   :   2   Min.   :    8   Min.   :   2   Min.   : 0.50  
##  1st Qu.: 300   1st Qu.:  810   1st Qu.: 285   1st Qu.: 2.56  
##  Median : 438   Median : 1168   Median : 416   Median : 3.51  
##  Mean   : 542   Mean   : 1425   Mean   : 504   Mean   : 3.87  
##  3rd Qu.: 658   3rd Qu.: 1743   3rd Qu.: 614   3rd Qu.: 4.72  
##  Max.   :6210   Max.   :16305   Max.   :5358   Max.   :15.00  
##  median_house_value  less1Hocean        inland          island     nearbay     
##  Min.   : 14999     Min.   :0.000   Min.   :0.000   Min.   :0   Min.   :0.000  
##  1st Qu.:120000     1st Qu.:0.000   1st Qu.:0.000   1st Qu.:0   1st Qu.:0.000  
##  Median :180950     Median :0.000   Median :0.000   Median :0   Median :0.000  
##  Mean   :207295     Mean   :0.447   Mean   :0.324   Mean   :0   Mean   :0.104  
##  3rd Qu.:265200     3rd Qu.:1.000   3rd Qu.:1.000   3rd Qu.:0   3rd Qu.:0.000  
##  Max.   :500001     Max.   :1.000   Max.   :1.000   Max.   :0   Max.   :1.000  
##    nearocean    
##  Min.   :0.000  
##  1st Qu.:0.000  
##  Median :0.000  
##  Mean   :0.125  
##  3rd Qu.:0.000  
##  Max.   :1.000
\end{verbatim}

\paragraph{Módulo 2.2.2: Transformação
Automática}\label{muxf3dulo-2.2.2-transformauxe7uxe3o-automuxe1tica}

\begin{Shaded}
\begin{Highlighting}[]
\NormalTok{auto\_one\_hot }\OtherTok{\textless{}{-}} \FunctionTok{model.matrix}\NormalTok{(}\SpecialCharTok{\textasciitilde{}}\NormalTok{ocean\_proximity }\SpecialCharTok{{-}} \DecValTok{1}\NormalTok{, }\AttributeTok{data =}\NormalTok{ train\_data)}
\NormalTok{auto\_one\_hot }\OtherTok{\textless{}{-}} \FunctionTok{cbind}\NormalTok{(train\_data, auto\_one\_hot)}

\NormalTok{auto\_one\_hot}\SpecialCharTok{$}\NormalTok{ocean\_proximity }\OtherTok{\textless{}{-}} \ConstantTok{NULL}

\FunctionTok{summary}\NormalTok{(auto\_one\_hot)}
\end{Highlighting}
\end{Shaded}

\begin{verbatim}
##    longitude       latitude    housing_median_age  total_rooms   
##  Min.   :-124   Min.   :32.5   Min.   : 1.0       Min.   :   15  
##  1st Qu.:-122   1st Qu.:33.9   1st Qu.:18.0       1st Qu.: 1461  
##  Median :-119   Median :34.3   Median :29.0       Median : 2130  
##  Mean   :-120   Mean   :35.6   Mean   :28.8       Mean   : 2641  
##  3rd Qu.:-118   3rd Qu.:37.7   3rd Qu.:37.0       3rd Qu.: 3131  
##  Max.   :-114   Max.   :42.0   Max.   :52.0       Max.   :37937  
##  total_bedrooms   population      households   median_income  
##  Min.   :   3   Min.   :    3   Min.   :   2   Min.   : 0.50  
##  1st Qu.: 297   1st Qu.:  793   1st Qu.: 281   1st Qu.: 2.56  
##  Median : 437   Median : 1169   Median : 411   Median : 3.54  
##  Mean   : 539   Mean   : 1430   Mean   : 500   Mean   : 3.87  
##  3rd Qu.: 647   3rd Qu.: 1720   3rd Qu.: 603   3rd Qu.: 4.76  
##  Max.   :6445   Max.   :35682   Max.   :6082   Max.   :15.00  
##  median_house_value ocean_proximity<1H OCEAN ocean_proximityINLAND
##  Min.   : 14999     Min.   :0.000            Min.   :0.000        
##  1st Qu.:119600     1st Qu.:0.000            1st Qu.:0.000        
##  Median :180300     Median :0.000            Median :0.000        
##  Mean   :206918     Mean   :0.437            Mean   :0.317        
##  3rd Qu.:265000     3rd Qu.:1.000            3rd Qu.:1.000        
##  Max.   :500001     Max.   :1.000            Max.   :1.000        
##  ocean_proximityISLAND ocean_proximityNEAR BAY ocean_proximityNEAR OCEAN
##  Min.   :0.000000      Min.   :0.000           Min.   :0.000            
##  1st Qu.:0.000000      1st Qu.:0.000           1st Qu.:0.000            
##  Median :0.000000      Median :0.000           Median :0.000            
##  Mean   :0.000244      Mean   :0.114           Mean   :0.132            
##  3rd Qu.:0.000000      3rd Qu.:0.000           3rd Qu.:0.000            
##  Max.   :1.000000      Max.   :1.000           Max.   :1.000
\end{verbatim}

\subsubsection{Módulo 2.3: Normalização dos
Dados}\label{muxf3dulo-2.3-normalizauxe7uxe3o-dos-dados}

Normalização é um processo que ajusta os valores das features para uma
escala comum, melhorando a performance dos modelos. Aqui, há exemplo de
dois métodos de normalização: Min-Max e Z-Normalização.

\paragraph{Módulo 2.3.1: Min-Max
Normalization}\label{muxf3dulo-2.3.1-min-max-normalization}

A formula do Min-Max é a seguinte:

\[x_{MinMax} = \frac{x - min(x)}{max(x) - min(x)}\]

\begin{Shaded}
\begin{Highlighting}[]
\CommentTok{\# Calculando os valores mínimos e máximos de cada variável}
\NormalTok{min\_features }\OtherTok{\textless{}{-}} \FunctionTok{apply}\NormalTok{(train\_encoded[, }\DecValTok{1}\SpecialCharTok{:}\DecValTok{8}\NormalTok{], }\DecValTok{2}\NormalTok{, min)}
\NormalTok{max\_features }\OtherTok{\textless{}{-}} \FunctionTok{apply}\NormalTok{(train\_encoded[, }\DecValTok{1}\SpecialCharTok{:}\DecValTok{8}\NormalTok{], }\DecValTok{2}\NormalTok{, max)}
\NormalTok{diff         }\OtherTok{\textless{}{-}}\NormalTok{ max\_features }\SpecialCharTok{{-}}\NormalTok{ min\_features}

\FunctionTok{print}\NormalTok{(}\StringTok{"Valores mínimos"}\NormalTok{)}
\end{Highlighting}
\end{Shaded}

\begin{verbatim}
## [1] "Valores mínimos"
\end{verbatim}

\begin{Shaded}
\begin{Highlighting}[]
\FunctionTok{print}\NormalTok{(min\_features)}
\end{Highlighting}
\end{Shaded}

\begin{verbatim}
##          longitude           latitude housing_median_age        total_rooms 
##             -124.3               32.5                1.0               15.0 
##     total_bedrooms         population         households      median_income 
##                3.0                3.0                2.0                0.5
\end{verbatim}

\begin{Shaded}
\begin{Highlighting}[]
\FunctionTok{print}\NormalTok{(}\StringTok{"Valores máximos"}\NormalTok{)}
\end{Highlighting}
\end{Shaded}

\begin{verbatim}
## [1] "Valores máximos"
\end{verbatim}

\begin{Shaded}
\begin{Highlighting}[]
\FunctionTok{print}\NormalTok{(max\_features)}
\end{Highlighting}
\end{Shaded}

\begin{verbatim}
##          longitude           latitude housing_median_age        total_rooms 
##               -114                 42                 52              37937 
##     total_bedrooms         population         households      median_income 
##               6445              35682               6082                 15
\end{verbatim}

\begin{Shaded}
\begin{Highlighting}[]
\FunctionTok{print}\NormalTok{(}\StringTok{"Diferença entre máximos e mínimos"}\NormalTok{)}
\end{Highlighting}
\end{Shaded}

\begin{verbatim}
## [1] "Diferença entre máximos e mínimos"
\end{verbatim}

\begin{Shaded}
\begin{Highlighting}[]
\FunctionTok{print}\NormalTok{(diff)}
\end{Highlighting}
\end{Shaded}

\begin{verbatim}
##          longitude           latitude housing_median_age        total_rooms 
##               10.0                9.4               51.0            37922.0 
##     total_bedrooms         population         households      median_income 
##             6442.0            35679.0             6080.0               14.5
\end{verbatim}

\begin{Shaded}
\begin{Highlighting}[]
\CommentTok{\# Aplicando a normalização Min{-}Max nos dados de treino e validação}
\NormalTok{train\_normalized }\OtherTok{\textless{}{-}}\NormalTok{ train\_encoded}
\NormalTok{valid\_normalized }\OtherTok{\textless{}{-}}\NormalTok{ valid\_encoded}

\NormalTok{train\_normalized[, }\DecValTok{1}\SpecialCharTok{:}\DecValTok{8}\NormalTok{] }\OtherTok{\textless{}{-}} \FunctionTok{sweep}\NormalTok{(train\_normalized[, }\DecValTok{1}\SpecialCharTok{:}\DecValTok{8}\NormalTok{], }\DecValTok{2}\NormalTok{, min\_features, }\StringTok{"{-}"}\NormalTok{)}
\NormalTok{train\_normalized[, }\DecValTok{1}\SpecialCharTok{:}\DecValTok{8}\NormalTok{] }\OtherTok{\textless{}{-}} \FunctionTok{sweep}\NormalTok{(train\_normalized[, }\DecValTok{1}\SpecialCharTok{:}\DecValTok{8}\NormalTok{], }\DecValTok{2}\NormalTok{, diff, }\StringTok{"/"}\NormalTok{)}

\NormalTok{valid\_normalized[, }\DecValTok{1}\SpecialCharTok{:}\DecValTok{8}\NormalTok{] }\OtherTok{\textless{}{-}} \FunctionTok{sweep}\NormalTok{(valid\_normalized[, }\DecValTok{1}\SpecialCharTok{:}\DecValTok{8}\NormalTok{], }\DecValTok{2}\NormalTok{, min\_features, }\StringTok{"{-}"}\NormalTok{)}
\NormalTok{valid\_normalized[, }\DecValTok{1}\SpecialCharTok{:}\DecValTok{8}\NormalTok{] }\OtherTok{\textless{}{-}} \FunctionTok{sweep}\NormalTok{(valid\_normalized[, }\DecValTok{1}\SpecialCharTok{:}\DecValTok{8}\NormalTok{], }\DecValTok{2}\NormalTok{, diff, }\StringTok{"/"}\NormalTok{)}
\end{Highlighting}
\end{Shaded}

\begin{Shaded}
\begin{Highlighting}[]
\FunctionTok{summary}\NormalTok{(train\_normalized)}
\end{Highlighting}
\end{Shaded}

\begin{verbatim}
##    longitude        latitude     housing_median_age  total_rooms    
##  Min.   :0.000   Min.   :0.000   Min.   :0.000      Min.   :0.0000  
##  1st Qu.:0.253   1st Qu.:0.148   1st Qu.:0.333      1st Qu.:0.0381  
##  Median :0.582   Median :0.182   Median :0.549      Median :0.0558  
##  Mean   :0.475   Mean   :0.329   Mean   :0.544      Mean   :0.0692  
##  3rd Qu.:0.632   3rd Qu.:0.551   3rd Qu.:0.706      3rd Qu.:0.0822  
##  Max.   :1.000   Max.   :1.000   Max.   :1.000      Max.   :1.0000  
##  total_bedrooms     population       households     median_income  
##  Min.   :0.0000   Min.   :0.0000   Min.   :0.0000   Min.   :0.000  
##  1st Qu.:0.0456   1st Qu.:0.0221   1st Qu.:0.0459   1st Qu.:0.142  
##  Median :0.0674   Median :0.0327   Median :0.0673   Median :0.210  
##  Mean   :0.0832   Mean   :0.0400   Mean   :0.0819   Mean   :0.232  
##  3rd Qu.:0.1000   3rd Qu.:0.0481   3rd Qu.:0.0988   3rd Qu.:0.294  
##  Max.   :1.0000   Max.   :1.0000   Max.   :1.0000   Max.   :1.000  
##  median_house_value  less1Hocean        inland          island        
##  Min.   : 14999     Min.   :0.000   Min.   :0.000   Min.   :0.000000  
##  1st Qu.:119600     1st Qu.:0.000   1st Qu.:0.000   1st Qu.:0.000000  
##  Median :180300     Median :0.000   Median :0.000   Median :0.000000  
##  Mean   :206918     Mean   :0.437   Mean   :0.317   Mean   :0.000244  
##  3rd Qu.:265000     3rd Qu.:1.000   3rd Qu.:1.000   3rd Qu.:0.000000  
##  Max.   :500001     Max.   :1.000   Max.   :1.000   Max.   :1.000000  
##     nearbay        nearocean    
##  Min.   :0.000   Min.   :0.000  
##  1st Qu.:0.000   1st Qu.:0.000  
##  Median :0.000   Median :0.000  
##  Mean   :0.114   Mean   :0.132  
##  3rd Qu.:0.000   3rd Qu.:0.000  
##  Max.   :1.000   Max.   :1.000
\end{verbatim}

\begin{Shaded}
\begin{Highlighting}[]
\FunctionTok{summary}\NormalTok{(valid\_normalized)}
\end{Highlighting}
\end{Shaded}

\begin{verbatim}
##    longitude        latitude        housing_median_age  total_rooms       
##  Min.   :0.012   Min.   :-0.00106   Min.   :0.000      Min.   :-0.000237  
##  1st Qu.:0.264   1st Qu.: 0.14574   1st Qu.:0.333      1st Qu.: 0.038236  
##  Median :0.585   Median : 0.18298   Median :0.529      Median : 0.056339  
##  Mean   :0.478   Mean   : 0.32698   Mean   :0.533      Mean   : 0.069430  
##  3rd Qu.:0.631   3rd Qu.: 0.54681   3rd Qu.:0.686      3rd Qu.: 0.083546  
##  Max.   :0.971   Max.   : 1.00000   Max.   :1.000      Max.   : 1.036470  
##  total_bedrooms        population        households     median_income  
##  Min.   :-0.000155   Min.   :0.00014   Min.   :0.0000   Min.   :0.000  
##  1st Qu.: 0.046104   1st Qu.:0.02261   1st Qu.:0.0465   1st Qu.:0.142  
##  Median : 0.067603   Median :0.03267   Median :0.0681   Median :0.208  
##  Mean   : 0.083603   Mean   :0.03985   Mean   :0.0826   Mean   :0.233  
##  3rd Qu.: 0.101677   3rd Qu.:0.04877   3rd Qu.:0.1007   3rd Qu.:0.291  
##  Max.   : 0.963521   Max.   :0.45691   Max.   :0.8809   Max.   :1.000  
##  median_house_value  less1Hocean        inland          island     nearbay     
##  Min.   : 14999     Min.   :0.000   Min.   :0.000   Min.   :0   Min.   :0.000  
##  1st Qu.:120000     1st Qu.:0.000   1st Qu.:0.000   1st Qu.:0   1st Qu.:0.000  
##  Median :180950     Median :0.000   Median :0.000   Median :0   Median :0.000  
##  Mean   :207295     Mean   :0.447   Mean   :0.324   Mean   :0   Mean   :0.104  
##  3rd Qu.:265200     3rd Qu.:1.000   3rd Qu.:1.000   3rd Qu.:0   3rd Qu.:0.000  
##  Max.   :500001     Max.   :1.000   Max.   :1.000   Max.   :0   Max.   :1.000  
##    nearocean    
##  Min.   :0.000  
##  1st Qu.:0.000  
##  Median :0.000  
##  Mean   :0.125  
##  3rd Qu.:0.000  
##  Max.   :1.000
\end{verbatim}

\begin{Shaded}
\begin{Highlighting}[]
\NormalTok{valid\_normalized[, }\DecValTok{1}\SpecialCharTok{:}\DecValTok{8}\NormalTok{] }\OtherTok{\textless{}{-}} \FunctionTok{as.data.frame}\NormalTok{(}\FunctionTok{lapply}\NormalTok{(valid\_normalized[, }\DecValTok{1}\SpecialCharTok{:}\DecValTok{8}\NormalTok{], }\ControlFlowTok{function}\NormalTok{(x) }\FunctionTok{pmax}\NormalTok{(}\DecValTok{0}\NormalTok{, }\FunctionTok{pmin}\NormalTok{(}\DecValTok{1}\NormalTok{, x))))}
\FunctionTok{summary}\NormalTok{(valid\_normalized)}
\end{Highlighting}
\end{Shaded}

\begin{verbatim}
##    longitude        latitude     housing_median_age  total_rooms    
##  Min.   :0.012   Min.   :0.000   Min.   :0.000      Min.   :0.0000  
##  1st Qu.:0.264   1st Qu.:0.146   1st Qu.:0.333      1st Qu.:0.0382  
##  Median :0.585   Median :0.183   Median :0.529      Median :0.0563  
##  Mean   :0.478   Mean   :0.327   Mean   :0.533      Mean   :0.0694  
##  3rd Qu.:0.631   3rd Qu.:0.547   3rd Qu.:0.686      3rd Qu.:0.0835  
##  Max.   :0.971   Max.   :1.000   Max.   :1.000      Max.   :1.0000  
##  total_bedrooms     population        households     median_income  
##  Min.   :0.0000   Min.   :0.00014   Min.   :0.0000   Min.   :0.000  
##  1st Qu.:0.0461   1st Qu.:0.02261   1st Qu.:0.0465   1st Qu.:0.142  
##  Median :0.0676   Median :0.03267   Median :0.0681   Median :0.208  
##  Mean   :0.0836   Mean   :0.03985   Mean   :0.0826   Mean   :0.233  
##  3rd Qu.:0.1017   3rd Qu.:0.04877   3rd Qu.:0.1007   3rd Qu.:0.291  
##  Max.   :0.9635   Max.   :0.45691   Max.   :0.8809   Max.   :1.000  
##  median_house_value  less1Hocean        inland          island     nearbay     
##  Min.   : 14999     Min.   :0.000   Min.   :0.000   Min.   :0   Min.   :0.000  
##  1st Qu.:120000     1st Qu.:0.000   1st Qu.:0.000   1st Qu.:0   1st Qu.:0.000  
##  Median :180950     Median :0.000   Median :0.000   Median :0   Median :0.000  
##  Mean   :207295     Mean   :0.447   Mean   :0.324   Mean   :0   Mean   :0.104  
##  3rd Qu.:265200     3rd Qu.:1.000   3rd Qu.:1.000   3rd Qu.:0   3rd Qu.:0.000  
##  Max.   :500001     Max.   :1.000   Max.   :1.000   Max.   :0   Max.   :1.000  
##    nearocean    
##  Min.   :0.000  
##  1st Qu.:0.000  
##  Median :0.000  
##  Mean   :0.125  
##  3rd Qu.:0.000  
##  Max.   :1.000
\end{verbatim}

\paragraph{Módulo 2.3.2:
Z-Normalization}\label{muxf3dulo-2.3.2-z-normalization}

A formula do Z-norm é a seguinte:

\[x_{Znorm} = \frac{x - mean(x)}{sd(x)}\]

\begin{Shaded}
\begin{Highlighting}[]
\CommentTok{\# Calculando a média e o desvio padrão para normalização}
\NormalTok{mean\_features }\OtherTok{\textless{}{-}} \FunctionTok{apply}\NormalTok{(train\_encoded[, }\DecValTok{1}\SpecialCharTok{:}\DecValTok{8}\NormalTok{], }\DecValTok{2}\NormalTok{, mean)}
\NormalTok{sd\_features   }\OtherTok{\textless{}{-}} \FunctionTok{apply}\NormalTok{(train\_encoded[, }\DecValTok{1}\SpecialCharTok{:}\DecValTok{8}\NormalTok{], }\DecValTok{2}\NormalTok{, sd)}

\FunctionTok{print}\NormalTok{(}\StringTok{"Médias das features"}\NormalTok{)}
\end{Highlighting}
\end{Shaded}

\begin{verbatim}
## [1] "Médias das features"
\end{verbatim}

\begin{Shaded}
\begin{Highlighting}[]
\FunctionTok{print}\NormalTok{(mean\_features)}
\end{Highlighting}
\end{Shaded}

\begin{verbatim}
##          longitude           latitude housing_median_age        total_rooms 
##            -119.58              35.64              28.76            2640.81 
##     total_bedrooms         population         households      median_income 
##             538.82            1429.55             500.01               3.87
\end{verbatim}

\begin{Shaded}
\begin{Highlighting}[]
\FunctionTok{print}\NormalTok{(}\StringTok{"Desvios padrão das features"}\NormalTok{)}
\end{Highlighting}
\end{Shaded}

\begin{verbatim}
## [1] "Desvios padrão das features"
\end{verbatim}

\begin{Shaded}
\begin{Highlighting}[]
\FunctionTok{print}\NormalTok{(sd\_features)}
\end{Highlighting}
\end{Shaded}

\begin{verbatim}
##          longitude           latitude housing_median_age        total_rooms 
##               2.01               2.14              12.62            2185.67 
##     total_bedrooms         population         households      median_income 
##             421.63            1150.94             382.11               1.89
\end{verbatim}

\begin{Shaded}
\begin{Highlighting}[]
\CommentTok{\# Aplicando a normalização Z{-}Score nos dados de treino}
\NormalTok{z\_norm\_data }\OtherTok{\textless{}{-}}\NormalTok{ train\_encoded}

\NormalTok{z\_norm\_data[, }\DecValTok{1}\SpecialCharTok{:}\DecValTok{8}\NormalTok{] }\OtherTok{\textless{}{-}} \FunctionTok{sweep}\NormalTok{(z\_norm\_data[, }\DecValTok{1}\SpecialCharTok{:}\DecValTok{8}\NormalTok{], }\DecValTok{2}\NormalTok{, mean\_features, }\StringTok{"{-}"}\NormalTok{)}
\NormalTok{z\_norm\_data[, }\DecValTok{1}\SpecialCharTok{:}\DecValTok{8}\NormalTok{] }\OtherTok{\textless{}{-}} \FunctionTok{sweep}\NormalTok{(z\_norm\_data[, }\DecValTok{1}\SpecialCharTok{:}\DecValTok{8}\NormalTok{], }\DecValTok{2}\NormalTok{, sd\_features, }\StringTok{"/"}\NormalTok{)}

\FunctionTok{summary}\NormalTok{(z\_norm\_data)}
\end{Highlighting}
\end{Shaded}

\begin{verbatim}
##    longitude         latitude      housing_median_age  total_rooms    
##  Min.   :-2.369   Min.   :-1.442   Min.   :-2.201     Min.   :-1.201  
##  1st Qu.:-1.108   1st Qu.:-0.793   1st Qu.:-0.853     1st Qu.:-0.540  
##  Median : 0.532   Median :-0.644   Median : 0.019     Median :-0.234  
##  Mean   : 0.000   Mean   : 0.000   Mean   : 0.000     Mean   : 0.000  
##  3rd Qu.: 0.785   3rd Qu.: 0.975   3rd Qu.: 0.653     3rd Qu.: 0.224  
##  Max.   : 2.618   Max.   : 2.944   Max.   : 1.842     Max.   :16.149  
##  total_bedrooms     population       households     median_income   
##  Min.   :-1.271   Min.   :-1.239   Min.   :-1.303   Min.   :-1.779  
##  1st Qu.:-0.574   1st Qu.:-0.553   1st Qu.:-0.573   1st Qu.:-0.692  
##  Median :-0.241   Median :-0.226   Median :-0.233   Median :-0.173  
##  Mean   : 0.000   Mean   : 0.000   Mean   : 0.000   Mean   : 0.000  
##  3rd Qu.: 0.257   3rd Qu.: 0.252   3rd Qu.: 0.270   3rd Qu.: 0.470  
##  Max.   :14.008   Max.   :29.760   Max.   :14.608   Max.   : 5.878  
##  median_house_value  less1Hocean        inland          island        
##  Min.   : 14999     Min.   :0.000   Min.   :0.000   Min.   :0.000000  
##  1st Qu.:119600     1st Qu.:0.000   1st Qu.:0.000   1st Qu.:0.000000  
##  Median :180300     Median :0.000   Median :0.000   Median :0.000000  
##  Mean   :206918     Mean   :0.437   Mean   :0.317   Mean   :0.000244  
##  3rd Qu.:265000     3rd Qu.:1.000   3rd Qu.:1.000   3rd Qu.:0.000000  
##  Max.   :500001     Max.   :1.000   Max.   :1.000   Max.   :1.000000  
##     nearbay        nearocean    
##  Min.   :0.000   Min.   :0.000  
##  1st Qu.:0.000   1st Qu.:0.000  
##  Median :0.000   Median :0.000  
##  Mean   :0.114   Mean   :0.132  
##  3rd Qu.:0.000   3rd Qu.:0.000  
##  Max.   :1.000   Max.   :1.000
\end{verbatim}

\subsubsection{Módulo 2.4: Análises
estastísticas}\label{muxf3dulo-2.4-anuxe1lises-estastuxedsticas}

Há diversas análises que podemos realizar. Aqui, apresentamos um exemplo
de correlação entre as features, o que pode ser útil para identificar
relações entre as variáveis e orientar a combinação de features no
modelo.

\begin{Shaded}
\begin{Highlighting}[]
\CommentTok{\# Calculando a matriz de correlação}
\NormalTok{correlation }\OtherTok{\textless{}{-}} \FunctionTok{cor}\NormalTok{(train\_data[, }\DecValTok{1}\SpecialCharTok{:}\DecValTok{8}\NormalTok{])}
\FunctionTok{print}\NormalTok{(correlation)}
\end{Highlighting}
\end{Shaded}

\begin{verbatim}
##                    longitude latitude housing_median_age total_rooms
## longitude             1.0000  -0.9252            -0.1165      0.0474
## latitude             -0.9252   1.0000             0.0185     -0.0381
## housing_median_age   -0.1165   0.0185             1.0000     -0.3611
## total_rooms           0.0474  -0.0381            -0.3611      1.0000
## total_bedrooms        0.0701  -0.0667            -0.3173      0.9309
## population            0.0981  -0.1070            -0.2917      0.8478
## households            0.0553  -0.0708            -0.3004      0.9175
## median_income        -0.0111  -0.0837            -0.1243      0.1976
##                    total_bedrooms population households median_income
## longitude                 0.07011    0.09814     0.0553      -0.01114
## latitude                 -0.06674   -0.10698    -0.0708      -0.08372
## housing_median_age       -0.31726   -0.29165    -0.3004      -0.12425
## total_rooms               0.93086    0.84776     0.9175       0.19760
## total_bedrooms            1.00000    0.86821     0.9783      -0.00942
## population                0.86821    1.00000     0.8989       0.00313
## households                0.97834    0.89891     1.0000       0.01266
## median_income            -0.00942    0.00313     0.0127       1.00000
\end{verbatim}

\begin{Shaded}
\begin{Highlighting}[]
\CommentTok{\# Visualizando a matriz de correlação com um heatmap}
\FunctionTok{par}\NormalTok{(}\AttributeTok{mar =} \FunctionTok{c}\NormalTok{(}\DecValTok{2}\NormalTok{, }\DecValTok{2}\NormalTok{, }\DecValTok{2}\NormalTok{, }\DecValTok{2}\NormalTok{))  }\CommentTok{\# Ajusta as margens do gráfico}
\FunctionTok{par}\NormalTok{(}\AttributeTok{cex =} \FloatTok{1.2}\NormalTok{)            }\CommentTok{\# Ajusta o tamanho do texto}

\FunctionTok{dev.new}\NormalTok{(}\AttributeTok{width =} \DecValTok{15}\NormalTok{, }\AttributeTok{height =} \DecValTok{15}\NormalTok{, }\AttributeTok{unit =} \StringTok{"in"}\NormalTok{, }\AttributeTok{noRStudioGD =} \ConstantTok{TRUE}\NormalTok{)}

\FunctionTok{corrplot}\NormalTok{(correlation, }\AttributeTok{method =} \StringTok{"color"}\NormalTok{, }\AttributeTok{type =} \StringTok{"upper"}\NormalTok{,}
         \AttributeTok{addCoef.col =} \StringTok{"black"}\NormalTok{, }\AttributeTok{tl.col =} \StringTok{"black"}\NormalTok{, }\AttributeTok{tl.srt =} \DecValTok{45}\NormalTok{, }\AttributeTok{number.cex =} \DecValTok{1}\NormalTok{)}
\end{Highlighting}
\end{Shaded}

\begin{center}\rule{0.5\linewidth}{0.5pt}\end{center}

\subsection{Módulo 3: Regressão
Linear}\label{muxf3dulo-3-regressuxe3o-linear}

Vamos trabalhar com regressão linear com múltiplas variáveis. Lembrando
que a fórmula é:

\[h_{\theta}(x) = \theta_0 + \theta_1x_1 + \theta_2x_2 + \dots + \theta_nx_n\]

Para avaliar o desempenho de novas técnicas de modelagem, é importante
ter um modelo baseline (de referência) para comparar se as novas
abordagens realmente melhoram os resultados. Neste Módulo, treinaremos
uma regressão linear simples, utilizando apenas as variáveis contínuas
presentes nos dados.

\subsubsection{Módulo 3.1: Treinamento do
Modelo}\label{muxf3dulo-3.1-treinamento-do-modelo}

Por padrão, a biblioteca \texttt{lm()} utiliza a equação normal para
ajustar os coeficientes do modelo. Caso o modelo se torne mais complexo
ou os dados sejam muito grandes, o R automaticamente recorre ao método
de descida de gradiente para otimizar os coeficientes.

\begin{Shaded}
\begin{Highlighting}[]
\CommentTok{\# Treinando o modelo de regressão linear com as variáveis contínuas}
\NormalTok{baseline }\OtherTok{\textless{}{-}} \FunctionTok{lm}\NormalTok{(}\AttributeTok{formula =}\NormalTok{ median\_house\_value }\SpecialCharTok{\textasciitilde{}}\NormalTok{ longitude }\SpecialCharTok{+}\NormalTok{ latitude}
               \SpecialCharTok{+}\NormalTok{ housing\_median\_age }\SpecialCharTok{+}\NormalTok{ total\_rooms }\SpecialCharTok{+}\NormalTok{ total\_bedrooms}
               \SpecialCharTok{+}\NormalTok{ population }\SpecialCharTok{+}\NormalTok{ households }\SpecialCharTok{+}\NormalTok{ median\_income, }
               \AttributeTok{data=}\NormalTok{train\_normalized)}

\CommentTok{\# Exibindo o resumo do modelo ajustado}
\FunctionTok{summary}\NormalTok{(baseline)}
\end{Highlighting}
\end{Shaded}

\begin{verbatim}
## 
## Call:
## lm(formula = median_house_value ~ longitude + latitude + housing_median_age + 
##     total_rooms + total_bedrooms + population + households + 
##     median_income, data = train_normalized)
## 
## Residuals:
##     Min      1Q  Median      3Q     Max 
## -382945  -43701  -11292   30209  712694 
## 
## Coefficients:
##                    Estimate Std. Error t value Pr(>|t|)    
## (Intercept)          354024       8389   42.20   <2e-16 ***
## longitude           -420680       9244  -45.51   <2e-16 ***
## latitude            -391010       8173  -47.84   <2e-16 ***
## housing_median_age    62799       2834   22.16   <2e-16 ***
## total_rooms         -388388      38750  -10.02   <2e-16 ***
## total_bedrooms       798615      56189   14.21   <2e-16 ***
## population         -1223436      46634  -26.23   <2e-16 ***
## households           221795      56615    3.92    9e-05 ***
## median_income        595214       6312   94.29   <2e-16 ***
## ---
## Signif. codes:  0 '***' 0.001 '**' 0.01 '*' 0.05 '.' 0.1 ' ' 1
## 
## Residual standard error: 69100 on 12259 degrees of freedom
## Multiple R-squared:  0.64,   Adjusted R-squared:  0.639 
## F-statistic: 2.72e+03 on 8 and 12259 DF,  p-value: <2e-16
\end{verbatim}

\subsubsection{Módulo 3.2: Predição}\label{muxf3dulo-3.2-prediuxe7uxe3o}

Uma vez que o modelo está treinado, podemos realizar previsões tanto
para os dados de treino quanto para os dados de validação.

\begin{Shaded}
\begin{Highlighting}[]
\CommentTok{\# Realizando previsões nos dados de treino e validação}
\NormalTok{train\_pred }\OtherTok{\textless{}{-}} \FunctionTok{predict}\NormalTok{(baseline, train\_normalized)}
\NormalTok{valid\_pred }\OtherTok{\textless{}{-}} \FunctionTok{predict}\NormalTok{(baseline, valid\_normalized)}
\end{Highlighting}
\end{Shaded}

\subsubsection{Módulo 3.3: Cálculo das Métricas de
Desempenho}\label{muxf3dulo-3.3-cuxe1lculo-das-muxe9tricas-de-desempenho}

A seguir, implementamos três métricas importantes para avaliar o
desempenho do modelo:

\begin{enumerate}
\def\labelenumi{\arabic{enumi}.}
\tightlist
\item
  Erro Absoluto Médio (MAE):
  \[MAE = \frac{1}{m}\sum^m_{i=1}|h_{\theta}(x^i)-y^i|\]
\item
  Erro Quadrático Médio (MSE):
  \[MSE = \frac{1}{m}\sum^m_{i=1}(h_{\theta}(x^i)-y^i)^2\]
\item
  Coeficiente de Determinação (R²):
  \[R^2 = 1 - \frac{RSS}{TSS} = 1 - \frac{\sum^m_{i=1}(h_{\theta}(x^i)-y^i)^2}{\sum^m_{i=1}(y^i - \bar{y})^2}\]
\end{enumerate}

\begin{Shaded}
\begin{Highlighting}[]
\CommentTok{\# Função para calcular o MAE}
\NormalTok{MAE }\OtherTok{\textless{}{-}} \ControlFlowTok{function}\NormalTok{(preds, labels) \{}
\NormalTok{  mae\_values }\OtherTok{\textless{}{-}} \FunctionTok{sum}\NormalTok{(}\FunctionTok{abs}\NormalTok{(preds }\SpecialCharTok{{-}}\NormalTok{ labels)) }\SpecialCharTok{/} \FunctionTok{length}\NormalTok{(preds)}
  \FunctionTok{return}\NormalTok{(mae\_values)}
\NormalTok{\}}

\CommentTok{\# Função para calcular o MSE}
\NormalTok{MSE }\OtherTok{\textless{}{-}} \ControlFlowTok{function}\NormalTok{(preds, labels) \{}
\NormalTok{  mse\_values }\OtherTok{\textless{}{-}} \FunctionTok{sum}\NormalTok{((preds }\SpecialCharTok{{-}}\NormalTok{ labels)}\SpecialCharTok{\^{}}\DecValTok{2}\NormalTok{) }\SpecialCharTok{/} \FunctionTok{length}\NormalTok{(preds)}
  \FunctionTok{return}\NormalTok{(mse\_values)}
\NormalTok{\}}

\CommentTok{\# Função para calcular o R²}
\NormalTok{R2 }\OtherTok{\textless{}{-}} \ControlFlowTok{function}\NormalTok{(pred, true) \{}
\NormalTok{  rss }\OtherTok{\textless{}{-}} \FunctionTok{sum}\NormalTok{((pred }\SpecialCharTok{{-}}\NormalTok{ true)}\SpecialCharTok{\^{}}\DecValTok{2}\NormalTok{)}
\NormalTok{  tss }\OtherTok{\textless{}{-}} \FunctionTok{sum}\NormalTok{((true }\SpecialCharTok{{-}} \FunctionTok{mean}\NormalTok{(true))}\SpecialCharTok{\^{}}\DecValTok{2}\NormalTok{)}
\NormalTok{  r2 }\OtherTok{\textless{}{-}} \DecValTok{1} \SpecialCharTok{{-}}\NormalTok{ rss }\SpecialCharTok{/}\NormalTok{ tss}
  \FunctionTok{return}\NormalTok{(r2)}
\NormalTok{\}}
\end{Highlighting}
\end{Shaded}

\subsubsection{Módulo 3.4: Avaliação do
Modelo}\label{muxf3dulo-3.4-avaliauxe7uxe3o-do-modelo}

\begin{Shaded}
\begin{Highlighting}[]
\CommentTok{\# Calculando as métricas para o conjunto de treino}
\NormalTok{mae\_train }\OtherTok{\textless{}{-}} \FunctionTok{MAE}\NormalTok{(train\_pred, train\_normalized}\SpecialCharTok{$}\NormalTok{median\_house\_value)}
\NormalTok{mse\_train }\OtherTok{\textless{}{-}} \FunctionTok{MSE}\NormalTok{(train\_pred, train\_normalized}\SpecialCharTok{$}\NormalTok{median\_house\_value)}
\NormalTok{r2\_train  }\OtherTok{\textless{}{-}} \FunctionTok{R2}\NormalTok{(train\_pred, train\_normalized}\SpecialCharTok{$}\NormalTok{median\_house\_value)}

\CommentTok{\# Calculando as métricas para o conjunto de validação}
\NormalTok{mae\_valid }\OtherTok{\textless{}{-}} \FunctionTok{MAE}\NormalTok{(valid\_pred, valid\_normalized}\SpecialCharTok{$}\NormalTok{median\_house\_value)}
\NormalTok{mse\_valid }\OtherTok{\textless{}{-}} \FunctionTok{MSE}\NormalTok{(valid\_pred, valid\_normalized}\SpecialCharTok{$}\NormalTok{median\_house\_value)}
\NormalTok{r2\_valid  }\OtherTok{\textless{}{-}} \FunctionTok{R2}\NormalTok{(valid\_pred, valid\_normalized}\SpecialCharTok{$}\NormalTok{median\_house\_value)}


\CommentTok{\# Organizando os resultados em um dataframe}
\NormalTok{results\_baseline }\OtherTok{\textless{}{-}} \FunctionTok{data.frame}\NormalTok{(}
  \AttributeTok{Metric =} \FunctionTok{c}\NormalTok{(}\StringTok{"MAE"}\NormalTok{, }\StringTok{"MSE"}\NormalTok{, }\StringTok{"R2"}\NormalTok{),}
  \AttributeTok{Train =} \FunctionTok{c}\NormalTok{(mae\_train, mse\_train, r2\_train),}
  \AttributeTok{Valid =} \FunctionTok{c}\NormalTok{(mae\_valid, mse\_valid, r2\_valid)}
\NormalTok{)}

\CommentTok{\# Formatando os resultados para melhor leitura}
\NormalTok{results\_baseline}\SpecialCharTok{$}\NormalTok{Train }\OtherTok{\textless{}{-}} \FunctionTok{as.numeric}\NormalTok{(results\_baseline}\SpecialCharTok{$}\NormalTok{Train)}
\NormalTok{results\_baseline}\SpecialCharTok{$}\NormalTok{Valid }\OtherTok{\textless{}{-}} \FunctionTok{as.numeric}\NormalTok{(results\_baseline}\SpecialCharTok{$}\NormalTok{Valid)}

\CommentTok{\# Evitando notação científica na exibição dos resultados}
\NormalTok{results\_baseline}\SpecialCharTok{$}\NormalTok{Train }\OtherTok{\textless{}{-}} \FunctionTok{format}\NormalTok{(results\_baseline}\SpecialCharTok{$}\NormalTok{Train, }\AttributeTok{scientific =} \ConstantTok{FALSE}\NormalTok{)}
\NormalTok{results\_baseline}\SpecialCharTok{$}\NormalTok{Valid }\OtherTok{\textless{}{-}} \FunctionTok{format}\NormalTok{(results\_baseline}\SpecialCharTok{$}\NormalTok{Valid, }\AttributeTok{scientific =} \ConstantTok{FALSE}\NormalTok{)}

\CommentTok{\# Exibindo os resultados}
\NormalTok{results\_baseline}
\end{Highlighting}
\end{Shaded}

\begin{verbatim}
##   Metric         Train          Valid
## 1    MAE      50608.57      51593.898
## 2    MSE 4765275558.77 5039378617.876
## 3     R2          0.64          0.625
\end{verbatim}

\begin{center}\rule{0.5\linewidth}{0.5pt}\end{center}

\subsection{Módulo 4: Combinação de
Features}\label{muxf3dulo-4-combinauxe7uxe3o-de-features}

Uma estratégia para melhorar o desempenho do modelo de regressão linear
é criar combinações de features. Isso pode ajudar a fornecer informações
mais explícitas ao modelo. Uma maneira eficaz de realizar essa
combinação é verificar as correlações entre as variáveis, combinando
aquelas com baixa correlação entre si e descartando as que têm alta
correlação, pois elas podem fornecer informações redundantes.

A seguir, apresentamos exemplos de como criar combinações de features no
modelo:

\subsubsection{Módulo 4.1: Definição das
Fórmulas}\label{muxf3dulo-4.1-definiuxe7uxe3o-das-fuxf3rmulas}

\begin{Shaded}
\begin{Highlighting}[]
\CommentTok{\# Fórmula com interações de ordem 2 entre as variáveis}
\NormalTok{f01 }\OtherTok{\textless{}{-}} \FunctionTok{formula}\NormalTok{(median\_house\_value }\SpecialCharTok{\textasciitilde{}}\NormalTok{ longitude }\SpecialCharTok{+}\NormalTok{ latitude }\SpecialCharTok{+}\NormalTok{ housing\_median\_age}
               \SpecialCharTok{+}\NormalTok{ total\_rooms }\SpecialCharTok{+}\NormalTok{ total\_bedrooms }\SpecialCharTok{+}\NormalTok{ population }\SpecialCharTok{+}\NormalTok{ households}
               \SpecialCharTok{+}\NormalTok{ median\_income }\SpecialCharTok{+}\NormalTok{ (longitude }\SpecialCharTok{+}\NormalTok{ latitude }\SpecialCharTok{+}\NormalTok{ housing\_median\_age}
               \SpecialCharTok{+}\NormalTok{ total\_rooms }\SpecialCharTok{+}\NormalTok{ total\_bedrooms }\SpecialCharTok{+}\NormalTok{ population }\SpecialCharTok{+}\NormalTok{ households}
               \SpecialCharTok{+}\NormalTok{ median\_income)}\SpecialCharTok{\^{}}\DecValTok{2}\NormalTok{)}

\CommentTok{\# Fórmula com interações de ordem 3 entre as variáveis}
\NormalTok{f02 }\OtherTok{\textless{}{-}} \FunctionTok{formula}\NormalTok{(median\_house\_value }\SpecialCharTok{\textasciitilde{}}\NormalTok{ longitude }\SpecialCharTok{+}\NormalTok{ latitude }\SpecialCharTok{+}\NormalTok{ housing\_median\_age}
               \SpecialCharTok{+}\NormalTok{ total\_rooms }\SpecialCharTok{+}\NormalTok{ total\_bedrooms }\SpecialCharTok{+}\NormalTok{ population }\SpecialCharTok{+}\NormalTok{ households}
               \SpecialCharTok{+}\NormalTok{ median\_income }\SpecialCharTok{+}\NormalTok{ (longitude }\SpecialCharTok{+}\NormalTok{ latitude }\SpecialCharTok{+}\NormalTok{ housing\_median\_age }
               \SpecialCharTok{+}\NormalTok{ total\_rooms }\SpecialCharTok{+}\NormalTok{ total\_bedrooms }\SpecialCharTok{+}\NormalTok{ population }\SpecialCharTok{+}\NormalTok{ households}
               \SpecialCharTok{+}\NormalTok{ median\_income)}\SpecialCharTok{\^{}}\DecValTok{3}\NormalTok{)}

\CommentTok{\# Fórmula com interações de ordem 4 entre as variáveis}
\NormalTok{f03 }\OtherTok{\textless{}{-}} \FunctionTok{formula}\NormalTok{(median\_house\_value }\SpecialCharTok{\textasciitilde{}}\NormalTok{ longitude }\SpecialCharTok{+}\NormalTok{ latitude }\SpecialCharTok{+}\NormalTok{ housing\_median\_age}
               \SpecialCharTok{+}\NormalTok{ total\_rooms }\SpecialCharTok{+}\NormalTok{ total\_bedrooms }\SpecialCharTok{+}\NormalTok{ population }\SpecialCharTok{+}\NormalTok{ households}
               \SpecialCharTok{+}\NormalTok{ median\_income }\SpecialCharTok{+}\NormalTok{ (longitude }\SpecialCharTok{+}\NormalTok{ latitude }\SpecialCharTok{+}\NormalTok{ housing\_median\_age}
               \SpecialCharTok{+}\NormalTok{ total\_rooms }\SpecialCharTok{+}\NormalTok{ total\_bedrooms }\SpecialCharTok{+}\NormalTok{ population }\SpecialCharTok{+}\NormalTok{ households}
               \SpecialCharTok{+}\NormalTok{ median\_income)}\SpecialCharTok{\^{}}\DecValTok{4}\NormalTok{)}
\end{Highlighting}
\end{Shaded}

\subsubsection{Módulo 4.2: Treinamento do
Modelo}\label{muxf3dulo-4.2-treinamento-do-modelo}

\begin{Shaded}
\begin{Highlighting}[]
\NormalTok{formulas }\OtherTok{\textless{}{-}} \FunctionTok{c}\NormalTok{(f01, f02, f03)}

\NormalTok{MaePerCombination }\OtherTok{\textless{}{-}} \FunctionTok{data.frame}\NormalTok{(}\AttributeTok{Formula  =} \FunctionTok{numeric}\NormalTok{(}\FunctionTok{length}\NormalTok{(formulas)), }
                                \AttributeTok{TrainMAE =} \FunctionTok{numeric}\NormalTok{(}\FunctionTok{length}\NormalTok{(formulas)),}
                                \AttributeTok{ValMAE   =} \FunctionTok{numeric}\NormalTok{(}\FunctionTok{length}\NormalTok{(formulas)))}

\NormalTok{i }\OtherTok{\textless{}{-}} \DecValTok{1}
\ControlFlowTok{for}\NormalTok{(f }\ControlFlowTok{in}\NormalTok{ formulas)\{}
  \CommentTok{\# Treinando o modelo com a fórmula atual}
\NormalTok{  model }\OtherTok{\textless{}{-}} \FunctionTok{lm}\NormalTok{(}\AttributeTok{formula =}\NormalTok{ f, }\AttributeTok{data =}\NormalTok{ train\_normalized)}
  
  \CommentTok{\# Realizando previsões para os dados de treino e validação}
\NormalTok{  train\_pred }\OtherTok{\textless{}{-}} \FunctionTok{predict}\NormalTok{(model, train\_normalized)}
\NormalTok{  valid\_pred }\OtherTok{\textless{}{-}} \FunctionTok{predict}\NormalTok{(model, valid\_normalized)}
  
  \CommentTok{\# Calculando o MAE para o conjunto de treino e validação}
\NormalTok{  mae\_train }\OtherTok{\textless{}{-}} \FunctionTok{MAE}\NormalTok{(train\_pred, train\_normalized}\SpecialCharTok{$}\NormalTok{median\_house\_value)}
\NormalTok{  mae\_val   }\OtherTok{\textless{}{-}} \FunctionTok{MAE}\NormalTok{(valid\_pred, valid\_normalized}\SpecialCharTok{$}\NormalTok{median\_house\_value)}
  
  \CommentTok{\# Armazenando os resultados}
\NormalTok{  MaePerCombination[i, ] }\OtherTok{\textless{}{-}} \FunctionTok{c}\NormalTok{(i, mae\_train, mae\_val)}
\NormalTok{  i }\OtherTok{\textless{}{-}}\NormalTok{ i }\SpecialCharTok{+} \DecValTok{1}
\NormalTok{\}}
\end{Highlighting}
\end{Shaded}

\subsubsection{Módulo 4.3: Curva de Viés e
Variância}\label{muxf3dulo-4.3-curva-de-viuxe9s-e-variuxe2ncia}

Uma maneira de avaliar a complexidade do modelo é por meio da curva de
viés e variância. Essa curva ajuda a entender se a complexidade do
modelo está sendo um fator positivo ou negativo para o desempenho,
fornecendo insights sobre o risco de underfitting ou overfitting.

\begin{Shaded}
\begin{Highlighting}[]
\NormalTok{MaePerCombination}
\end{Highlighting}
\end{Shaded}

\begin{verbatim}
##   Formula TrainMAE ValMAE
## 1       1    46534  47783
## 2       2    44686  46023
## 3       3    43782  45618
\end{verbatim}

\begin{Shaded}
\begin{Highlighting}[]
\NormalTok{MaePerCombinationMelt }\OtherTok{\textless{}{-}} \FunctionTok{melt}\NormalTok{(MaePerCombination, }\AttributeTok{id =} \StringTok{"Formula"}\NormalTok{)}

\CommentTok{\# Criando o gráfico da curva de viés/variância}
\NormalTok{p }\OtherTok{\textless{}{-}} \FunctionTok{ggplot}\NormalTok{(}\AttributeTok{data =}\NormalTok{ MaePerCombinationMelt, }\FunctionTok{aes}\NormalTok{(}\AttributeTok{x =}\NormalTok{ Formula, }\AttributeTok{y =}\NormalTok{ value, }\AttributeTok{colour =}\NormalTok{ variable)) }\SpecialCharTok{+} 
  \FunctionTok{geom\_line}\NormalTok{() }\SpecialCharTok{+} 
  \FunctionTok{geom\_point}\NormalTok{() }\SpecialCharTok{+} 
  \FunctionTok{ggtitle}\NormalTok{(}\StringTok{"Curva Viés/Variância"}\NormalTok{) }\SpecialCharTok{+} 
  \FunctionTok{ylab}\NormalTok{(}\StringTok{"MAE"}\NormalTok{) }\SpecialCharTok{+} 
  \FunctionTok{scale\_x\_discrete}\NormalTok{(}\AttributeTok{name =} \StringTok{"Fórmula"}\NormalTok{, }
                   \AttributeTok{limits =} \FunctionTok{as.character}\NormalTok{(}\DecValTok{1}\SpecialCharTok{:}\FunctionTok{length}\NormalTok{(formulas)))}

\NormalTok{p }\SpecialCharTok{+} \FunctionTok{theme}\NormalTok{(}\AttributeTok{legend.position =} \FunctionTok{c}\NormalTok{(}\FloatTok{0.7}\NormalTok{, }\FloatTok{0.85}\NormalTok{), }\AttributeTok{legend.title =} \FunctionTok{element\_blank}\NormalTok{())}
\end{Highlighting}
\end{Shaded}

\pandocbounded{\includegraphics[keepaspectratio]{Ex01_files/figure-latex/unnamed-chunk-24-1.pdf}}

\subsection{Módulo 5: Ajuda de IA}\label{muxf3dulo-5-ajuda-de-ia}

Uma abordagem interessante para criar combinações de features é utilizar
ferramentas de IA generativa, como o ChatGPT. Abaixo está um exemplo de
como um prompt foi usado para gerar combinações não-lineares de
features, com o objetivo de melhorar o modelo preditivo para estimar o
valor do imóvel.

O prompt enviado ao ChatGPT e Maritaka AI foi o seguinte:

\begin{itemize}
\tightlist
\item
  \texttt{Prompt:\ Considere\ as\ seguintes\ features\ em\ minha\ base\ de\ dados\ sobre\ imóveis,\ com\ a\ qual\ devemos\ prever\ o\ valor\ do\ imóvel\ (median\_house\_value):\ longitude,\ latitude,\ housing\_median\_age,\ total\_rooms,\ total\_bedrooms,\ population,\ households,\ median\_income.\ Gere\ novas\ features\ por\ meio\ da\ combinação\ não-linear\ das\ features\ a\ cima\ em\ R}
\end{itemize}

\begin{Shaded}
\begin{Highlighting}[]
\DocumentationTok{\#\# ChatGPT}
\NormalTok{f01 }\OtherTok{\textless{}{-}} \FunctionTok{formula}\NormalTok{(median\_house\_value }\SpecialCharTok{\textasciitilde{}}\NormalTok{ longitude }\SpecialCharTok{+}\NormalTok{ latitude }\SpecialCharTok{+}\NormalTok{ housing\_median\_age }
               \SpecialCharTok{+}\NormalTok{ total\_rooms }\SpecialCharTok{+}\NormalTok{ total\_bedrooms }\SpecialCharTok{+}\NormalTok{ population }\SpecialCharTok{+}\NormalTok{ households}
               \SpecialCharTok{+}\NormalTok{ median\_income }\SpecialCharTok{+}\NormalTok{ longitude}\SpecialCharTok{*}\NormalTok{latitude)}

\NormalTok{f02 }\OtherTok{\textless{}{-}} \FunctionTok{formula}\NormalTok{(median\_house\_value }\SpecialCharTok{\textasciitilde{}}\NormalTok{ longitude }\SpecialCharTok{+}\NormalTok{ latitude }\SpecialCharTok{+}\NormalTok{ housing\_median\_age}
               \SpecialCharTok{+}\NormalTok{ total\_rooms }\SpecialCharTok{+}\NormalTok{ total\_bedrooms }\SpecialCharTok{+}\NormalTok{ population }\SpecialCharTok{+}\NormalTok{ households}
               \SpecialCharTok{+}\NormalTok{ median\_income }\SpecialCharTok{+}\NormalTok{ total\_rooms}\SpecialCharTok{/}\NormalTok{households }\SpecialCharTok{+}\NormalTok{ total\_bedrooms}\SpecialCharTok{/}\NormalTok{total\_rooms)}

\NormalTok{f03 }\OtherTok{\textless{}{-}} \FunctionTok{formula}\NormalTok{(median\_house\_value }\SpecialCharTok{\textasciitilde{}}\NormalTok{ longitude }\SpecialCharTok{+}\NormalTok{ latitude }\SpecialCharTok{+}\NormalTok{ housing\_median\_age}
               \SpecialCharTok{+}\NormalTok{ total\_rooms }\SpecialCharTok{+}\NormalTok{ total\_bedrooms }\SpecialCharTok{+}\NormalTok{ population }\SpecialCharTok{+}\NormalTok{ households}
               \SpecialCharTok{+}\NormalTok{ median\_income }\SpecialCharTok{+}\NormalTok{ longitude}\SpecialCharTok{*}\NormalTok{latitude }\SpecialCharTok{+}\NormalTok{ total\_rooms}\SpecialCharTok{/}\NormalTok{households}
               \SpecialCharTok{+}\NormalTok{ total\_bedrooms}\SpecialCharTok{/}\NormalTok{total\_rooms)}

\DocumentationTok{\#\# Maritaca AI}
\NormalTok{f04 }\OtherTok{\textless{}{-}} \FunctionTok{formula}\NormalTok{(median\_house\_value }\SpecialCharTok{\textasciitilde{}}\NormalTok{ longitude }\SpecialCharTok{+}\NormalTok{ latitude }\SpecialCharTok{+}\NormalTok{ housing\_median\_age }\SpecialCharTok{+}
\NormalTok{               total\_rooms }\SpecialCharTok{+}\NormalTok{ total\_bedrooms }\SpecialCharTok{+}\NormalTok{ population }\SpecialCharTok{+}\NormalTok{ households }\SpecialCharTok{+}
\NormalTok{               median\_income }\SpecialCharTok{+}\NormalTok{ housing\_median\_age }\SpecialCharTok{*}\NormalTok{ median\_income)}

\NormalTok{f05 }\OtherTok{\textless{}{-}} \FunctionTok{formula}\NormalTok{(median\_house\_value }\SpecialCharTok{\textasciitilde{}}\NormalTok{ longitude }\SpecialCharTok{+}\NormalTok{ latitude }\SpecialCharTok{+}\NormalTok{ housing\_median\_age }\SpecialCharTok{+}
\NormalTok{               total\_rooms }\SpecialCharTok{+}\NormalTok{ total\_bedrooms }\SpecialCharTok{+}\NormalTok{ population }\SpecialCharTok{+}\NormalTok{ households }\SpecialCharTok{+}
\NormalTok{               median\_income }\SpecialCharTok{+}\NormalTok{ longitude }\SpecialCharTok{*}\NormalTok{ latitude }\SpecialCharTok{+}\NormalTok{ population}\SpecialCharTok{/}\NormalTok{total\_rooms)}

\NormalTok{f06 }\OtherTok{\textless{}{-}} \FunctionTok{formula}\NormalTok{(median\_house\_value }\SpecialCharTok{\textasciitilde{}}\NormalTok{ longitude }\SpecialCharTok{+}\NormalTok{ latitude }\SpecialCharTok{+}\NormalTok{ housing\_median\_age }\SpecialCharTok{+}
\NormalTok{               total\_rooms }\SpecialCharTok{+}\NormalTok{ total\_bedrooms }\SpecialCharTok{+}\NormalTok{ population }\SpecialCharTok{+}\NormalTok{ households }\SpecialCharTok{+}
\NormalTok{               median\_income }\SpecialCharTok{+}\NormalTok{ longitude }\SpecialCharTok{*}\NormalTok{ latitude }\SpecialCharTok{*}\NormalTok{ median\_income }\SpecialCharTok{+}
\NormalTok{               population}\SpecialCharTok{/}\NormalTok{households)}
\end{Highlighting}
\end{Shaded}

\subsubsection{Módulo 5.1: Treinamento}\label{muxf3dulo-5.1-treinamento}

\begin{Shaded}
\begin{Highlighting}[]
\NormalTok{formulas }\OtherTok{\textless{}{-}} \FunctionTok{c}\NormalTok{(f01, f02, f03, f04, f05, f06)}

\NormalTok{MaePerCombinationAI }\OtherTok{\textless{}{-}} \FunctionTok{data.frame}\NormalTok{(}\AttributeTok{combination =} \FunctionTok{numeric}\NormalTok{(}\FunctionTok{length}\NormalTok{(formulas)), }
                                       \AttributeTok{TrainMAE    =} \FunctionTok{numeric}\NormalTok{(}\FunctionTok{length}\NormalTok{(formulas)),}
                                       \AttributeTok{ValMAE      =} \FunctionTok{numeric}\NormalTok{(}\FunctionTok{length}\NormalTok{(formulas)))}

\NormalTok{i }\OtherTok{\textless{}{-}} \DecValTok{1}
\ControlFlowTok{for}\NormalTok{(f }\ControlFlowTok{in}\NormalTok{ formulas)\{}
  \CommentTok{\# Treinando o modelo com a fórmula atual}
\NormalTok{  model }\OtherTok{\textless{}{-}} \FunctionTok{lm}\NormalTok{(}\AttributeTok{formula =}\NormalTok{ f, }\AttributeTok{data =}\NormalTok{ train\_normalized)}
  
  \CommentTok{\# Realizando previsões para os dados de treino e validação}
\NormalTok{  valPred   }\OtherTok{\textless{}{-}} \FunctionTok{predict}\NormalTok{(model, valid\_normalized)}
\NormalTok{  trainPred }\OtherTok{\textless{}{-}} \FunctionTok{predict}\NormalTok{(model, train\_normalized)}
  
  \CommentTok{\# Calculando o MAE para o conjunto de treino e validação}
\NormalTok{  mae\_train }\OtherTok{\textless{}{-}} \FunctionTok{MAE}\NormalTok{(trainPred, train\_normalized}\SpecialCharTok{$}\NormalTok{median\_house\_value)}
\NormalTok{  mae\_val   }\OtherTok{\textless{}{-}} \FunctionTok{MAE}\NormalTok{(valPred, valid\_normalized}\SpecialCharTok{$}\NormalTok{median\_house\_value)}
  
  \CommentTok{\# Armazenando os resultados}
\NormalTok{  MaePerCombinationAI[i, ] }\OtherTok{\textless{}{-}} \FunctionTok{c}\NormalTok{(i, mae\_train, mae\_val)}
\NormalTok{  i }\OtherTok{\textless{}{-}}\NormalTok{ i }\SpecialCharTok{+} \DecValTok{1}
\NormalTok{\}}

\CommentTok{\# Exibindo o resumo do modelo final}
\FunctionTok{summary}\NormalTok{(model)}
\end{Highlighting}
\end{Shaded}

\begin{verbatim}
## 
## Call:
## lm(formula = f, data = train_normalized)
## 
## Residuals:
##     Min      1Q  Median      3Q     Max 
## -383541  -42693  -11357   30110  527680 
## 
## Coefficients:
##                                  Estimate Std. Error t value Pr(>|t|)    
## (Intercept)                        247531      15042   16.46  < 2e-16 ***
## longitude                         -280269      21194  -13.22  < 2e-16 ***
## latitude                          -270650      21997  -12.30  < 2e-16 ***
## housing_median_age                  57569       2845   20.23  < 2e-16 ***
## total_rooms                       -382189      38736   -9.87  < 2e-16 ***
## total_bedrooms                     762723      55942   13.63  < 2e-16 ***
## population                       -1378108      54034  -25.50  < 2e-16 ***
## households                         242566      56311    4.31  1.7e-05 ***
## median_income                     1289100      67112   19.21  < 2e-16 ***
## longitude:latitude                 102149      41585    2.46    0.014 *  
## longitude:median_income           -891652      99083   -9.00  < 2e-16 ***
## latitude:median_income            -838196     109887   -7.63  2.6e-14 ***
## population:households              435410      72033    6.04  1.5e-09 ***
## longitude:latitude:median_income  -190996     208398   -0.92    0.359    
## ---
## Signif. codes:  0 '***' 0.001 '**' 0.01 '*' 0.05 '.' 0.1 ' ' 1
## 
## Residual standard error: 68500 on 12254 degrees of freedom
## Multiple R-squared:  0.645,  Adjusted R-squared:  0.645 
## F-statistic: 1.72e+03 on 13 and 12254 DF,  p-value: <2e-16
\end{verbatim}

\begin{Shaded}
\begin{Highlighting}[]
\NormalTok{MaePerCombinationAI}
\end{Highlighting}
\end{Shaded}

\begin{verbatim}
##   combination TrainMAE ValMAE
## 1           1    50613  51602
## 2           2    50458  51435
## 3           3    50459  51439
## 4           4    50352  51564
## 5           5    50577  51690
## 6           6    49992  51253
\end{verbatim}

\begin{Shaded}
\begin{Highlighting}[]
\NormalTok{MaePerCombinationAIMelt }\OtherTok{\textless{}{-}} \FunctionTok{melt}\NormalTok{(MaePerCombinationAI, }\AttributeTok{id =} \StringTok{"combination"}\NormalTok{)}

\CommentTok{\# Criando o gráfico de linha para a curva de viés e variância}
\NormalTok{p }\OtherTok{\textless{}{-}} \FunctionTok{ggplot}\NormalTok{(}\AttributeTok{data =}\NormalTok{ MaePerCombinationAIMelt, }\FunctionTok{aes}\NormalTok{(}\AttributeTok{x =}\NormalTok{ combination, }\AttributeTok{y =}\NormalTok{ value, }\AttributeTok{colour =}\NormalTok{ variable)) }\SpecialCharTok{+} 
  \FunctionTok{geom\_line}\NormalTok{() }\SpecialCharTok{+} 
  \FunctionTok{geom\_point}\NormalTok{()}

\NormalTok{p }\OtherTok{\textless{}{-}}\NormalTok{ p }\SpecialCharTok{+} \FunctionTok{ggtitle}\NormalTok{(}\StringTok{"Curva Vies/Variância"}\NormalTok{) }\SpecialCharTok{+} 
  \FunctionTok{ylab}\NormalTok{(}\StringTok{"MAE"}\NormalTok{) }\SpecialCharTok{+} 
  \FunctionTok{scale\_x\_discrete}\NormalTok{(}\AttributeTok{name =} \StringTok{"Combination"}\NormalTok{, }\AttributeTok{limits =} \FunctionTok{as.character}\NormalTok{(}\DecValTok{1}\SpecialCharTok{:}\FunctionTok{length}\NormalTok{(formulas)))}

\NormalTok{p }\SpecialCharTok{+} \FunctionTok{theme}\NormalTok{(}\AttributeTok{legend.position =} \FunctionTok{c}\NormalTok{(}\FloatTok{0.75}\NormalTok{, }\FloatTok{0.40}\NormalTok{), }\AttributeTok{legend.title =} \FunctionTok{element\_blank}\NormalTok{())}
\end{Highlighting}
\end{Shaded}

\pandocbounded{\includegraphics[keepaspectratio]{Ex01_files/figure-latex/unnamed-chunk-27-1.pdf}}

\begin{center}\rule{0.5\linewidth}{0.5pt}\end{center}

\subsection{Módulo 6: Regressão
Polinomial}\label{muxf3dulo-6-regressuxe3o-polinomial}

Outra forma para aumentar a complexidade do modelo é utilizar regressão
polinomial. Isso pode melhorar o ajuste ao dado, especialmente quando as
relações entre as variáveis não são lineares.

\subsubsection{Módulo 6.1: Criando as Fórmulas
Manualmente}\label{muxf3dulo-6.1-criando-as-fuxf3rmulas-manualmente}

\begin{Shaded}
\begin{Highlighting}[]
\NormalTok{f01 }\OtherTok{\textless{}{-}} \FunctionTok{formula}\NormalTok{(median\_house\_value }\SpecialCharTok{\textasciitilde{}}\NormalTok{ longitude }\SpecialCharTok{+}\NormalTok{ latitude }\SpecialCharTok{+}\NormalTok{ housing\_median\_age}
               \SpecialCharTok{+}\NormalTok{ total\_rooms }\SpecialCharTok{+}\NormalTok{ total\_bedrooms }\SpecialCharTok{+}\NormalTok{ population }\SpecialCharTok{+}\NormalTok{ households}
               \SpecialCharTok{+}\NormalTok{ median\_income, }\AttributeTok{data=}\NormalTok{train\_normalized)}

\NormalTok{f02 }\OtherTok{\textless{}{-}} \FunctionTok{formula}\NormalTok{(median\_house\_value }\SpecialCharTok{\textasciitilde{}}\NormalTok{ longitude }\SpecialCharTok{+}\NormalTok{ latitude }\SpecialCharTok{+}\NormalTok{ housing\_median\_age}
               \SpecialCharTok{+}\NormalTok{ total\_rooms }\SpecialCharTok{+}\NormalTok{ total\_bedrooms }\SpecialCharTok{+}\NormalTok{ population }\SpecialCharTok{+}\NormalTok{ households}
               \SpecialCharTok{+}\NormalTok{ median\_income }\SpecialCharTok{+} \FunctionTok{I}\NormalTok{(longitude}\SpecialCharTok{\^{}}\DecValTok{2}\NormalTok{) }\SpecialCharTok{+} \FunctionTok{I}\NormalTok{(latitude}\SpecialCharTok{\^{}}\DecValTok{2}\NormalTok{)}
               \SpecialCharTok{+} \FunctionTok{I}\NormalTok{(housing\_median\_age}\SpecialCharTok{\^{}}\DecValTok{2}\NormalTok{) }\SpecialCharTok{+} \FunctionTok{I}\NormalTok{(total\_rooms}\SpecialCharTok{\^{}}\DecValTok{2}\NormalTok{) }\SpecialCharTok{+} \FunctionTok{I}\NormalTok{(total\_bedrooms}\SpecialCharTok{\^{}}\DecValTok{2}\NormalTok{)}
               \SpecialCharTok{+} \FunctionTok{I}\NormalTok{(population}\SpecialCharTok{\^{}}\DecValTok{2}\NormalTok{) }\SpecialCharTok{+} \FunctionTok{I}\NormalTok{(households}\SpecialCharTok{\^{}}\DecValTok{2}\NormalTok{) }\SpecialCharTok{+} \FunctionTok{I}\NormalTok{(median\_income}\SpecialCharTok{\^{}}\DecValTok{2}\NormalTok{), }
               \AttributeTok{data=}\NormalTok{train\_normalized)}

\NormalTok{f03 }\OtherTok{\textless{}{-}} \FunctionTok{formula}\NormalTok{(median\_house\_value }\SpecialCharTok{\textasciitilde{}}\NormalTok{ longitude }\SpecialCharTok{+}\NormalTok{ latitude }\SpecialCharTok{+}\NormalTok{ housing\_median\_age}
               \SpecialCharTok{+}\NormalTok{ total\_rooms }\SpecialCharTok{+}\NormalTok{ total\_bedrooms }\SpecialCharTok{+}\NormalTok{ population }\SpecialCharTok{+}\NormalTok{ households}
               \SpecialCharTok{+}\NormalTok{ median\_income }\SpecialCharTok{+} \FunctionTok{I}\NormalTok{(longitude}\SpecialCharTok{\^{}}\DecValTok{2}\NormalTok{) }\SpecialCharTok{+} \FunctionTok{I}\NormalTok{(latitude}\SpecialCharTok{\^{}}\DecValTok{2}\NormalTok{)}
               \SpecialCharTok{+} \FunctionTok{I}\NormalTok{(housing\_median\_age}\SpecialCharTok{\^{}}\DecValTok{2}\NormalTok{) }\SpecialCharTok{+} \FunctionTok{I}\NormalTok{(total\_rooms}\SpecialCharTok{\^{}}\DecValTok{2}\NormalTok{) }\SpecialCharTok{+} \FunctionTok{I}\NormalTok{(total\_bedrooms}\SpecialCharTok{\^{}}\DecValTok{2}\NormalTok{)}
               \SpecialCharTok{+} \FunctionTok{I}\NormalTok{(population}\SpecialCharTok{\^{}}\DecValTok{2}\NormalTok{) }\SpecialCharTok{+} \FunctionTok{I}\NormalTok{(households}\SpecialCharTok{\^{}}\DecValTok{2}\NormalTok{) }\SpecialCharTok{+} \FunctionTok{I}\NormalTok{(median\_income}\SpecialCharTok{\^{}}\DecValTok{2}\NormalTok{)}
               \SpecialCharTok{+} \FunctionTok{I}\NormalTok{(longitude}\SpecialCharTok{\^{}}\DecValTok{3}\NormalTok{)  }\SpecialCharTok{+} \FunctionTok{I}\NormalTok{(latitude}\SpecialCharTok{\^{}}\DecValTok{3}\NormalTok{) }\SpecialCharTok{+} \FunctionTok{I}\NormalTok{(housing\_median\_age}\SpecialCharTok{\^{}}\DecValTok{3}\NormalTok{)}
               \SpecialCharTok{+} \FunctionTok{I}\NormalTok{(total\_rooms}\SpecialCharTok{\^{}}\DecValTok{3}\NormalTok{) }\SpecialCharTok{+} \FunctionTok{I}\NormalTok{(total\_bedrooms}\SpecialCharTok{\^{}}\DecValTok{3}\NormalTok{) }\SpecialCharTok{+} \FunctionTok{I}\NormalTok{(population}\SpecialCharTok{\^{}}\DecValTok{3}\NormalTok{)}
               \SpecialCharTok{+} \FunctionTok{I}\NormalTok{(households}\SpecialCharTok{\^{}}\DecValTok{3}\NormalTok{) }\SpecialCharTok{+} \FunctionTok{I}\NormalTok{(median\_income}\SpecialCharTok{\^{}}\DecValTok{3}\NormalTok{), }\AttributeTok{data=}\NormalTok{train\_normalized)}

\NormalTok{f04 }\OtherTok{\textless{}{-}} \FunctionTok{formula}\NormalTok{(median\_house\_value }\SpecialCharTok{\textasciitilde{}}\NormalTok{ longitude }\SpecialCharTok{+}\NormalTok{ latitude }\SpecialCharTok{+}\NormalTok{ housing\_median\_age}
               \SpecialCharTok{+}\NormalTok{ total\_rooms }\SpecialCharTok{+}\NormalTok{ total\_bedrooms }\SpecialCharTok{+}\NormalTok{ population }\SpecialCharTok{+}\NormalTok{ households}
               \SpecialCharTok{+}\NormalTok{ median\_income }\SpecialCharTok{+} \FunctionTok{I}\NormalTok{(longitude}\SpecialCharTok{\^{}}\DecValTok{2}\NormalTok{) }\SpecialCharTok{+} \FunctionTok{I}\NormalTok{(latitude}\SpecialCharTok{\^{}}\DecValTok{2}\NormalTok{)}
               \SpecialCharTok{+} \FunctionTok{I}\NormalTok{(housing\_median\_age}\SpecialCharTok{\^{}}\DecValTok{2}\NormalTok{) }\SpecialCharTok{+} \FunctionTok{I}\NormalTok{(total\_rooms}\SpecialCharTok{\^{}}\DecValTok{2}\NormalTok{) }\SpecialCharTok{+} \FunctionTok{I}\NormalTok{(total\_bedrooms}\SpecialCharTok{\^{}}\DecValTok{2}\NormalTok{)}
               \SpecialCharTok{+} \FunctionTok{I}\NormalTok{(population}\SpecialCharTok{\^{}}\DecValTok{2}\NormalTok{) }\SpecialCharTok{+} \FunctionTok{I}\NormalTok{(households}\SpecialCharTok{\^{}}\DecValTok{2}\NormalTok{) }\SpecialCharTok{+} \FunctionTok{I}\NormalTok{(median\_income}\SpecialCharTok{\^{}}\DecValTok{2}\NormalTok{)}
               \SpecialCharTok{+} \FunctionTok{I}\NormalTok{(longitude}\SpecialCharTok{\^{}}\DecValTok{3}\NormalTok{) }\SpecialCharTok{+} \FunctionTok{I}\NormalTok{(latitude}\SpecialCharTok{\^{}}\DecValTok{3}\NormalTok{) }\SpecialCharTok{+} \FunctionTok{I}\NormalTok{(housing\_median\_age}\SpecialCharTok{\^{}}\DecValTok{3}\NormalTok{)}
               \SpecialCharTok{+} \FunctionTok{I}\NormalTok{(total\_rooms}\SpecialCharTok{\^{}}\DecValTok{3}\NormalTok{) }\SpecialCharTok{+} \FunctionTok{I}\NormalTok{(total\_bedrooms}\SpecialCharTok{\^{}}\DecValTok{3}\NormalTok{) }\SpecialCharTok{+} \FunctionTok{I}\NormalTok{(population}\SpecialCharTok{\^{}}\DecValTok{3}\NormalTok{)}
               \SpecialCharTok{+} \FunctionTok{I}\NormalTok{(households}\SpecialCharTok{\^{}}\DecValTok{3}\NormalTok{) }\SpecialCharTok{+} \FunctionTok{I}\NormalTok{(median\_income}\SpecialCharTok{\^{}}\DecValTok{3}\NormalTok{) }\SpecialCharTok{+} \FunctionTok{I}\NormalTok{(longitude}\SpecialCharTok{\^{}}\DecValTok{4}\NormalTok{)}
               \SpecialCharTok{+} \FunctionTok{I}\NormalTok{(latitude}\SpecialCharTok{\^{}}\DecValTok{4}\NormalTok{) }\SpecialCharTok{+} \FunctionTok{I}\NormalTok{(housing\_median\_age}\SpecialCharTok{\^{}}\DecValTok{4}\NormalTok{) }\SpecialCharTok{+} \FunctionTok{I}\NormalTok{(total\_rooms}\SpecialCharTok{\^{}}\DecValTok{4}\NormalTok{)}
               \SpecialCharTok{+} \FunctionTok{I}\NormalTok{(total\_bedrooms}\SpecialCharTok{\^{}}\DecValTok{4}\NormalTok{) }\SpecialCharTok{+} \FunctionTok{I}\NormalTok{(population}\SpecialCharTok{\^{}}\DecValTok{4}\NormalTok{) }\SpecialCharTok{+} \FunctionTok{I}\NormalTok{(households}\SpecialCharTok{\^{}}\DecValTok{4}\NormalTok{)}
               \SpecialCharTok{+} \FunctionTok{I}\NormalTok{(median\_income}\SpecialCharTok{\^{}}\DecValTok{4}\NormalTok{), }\AttributeTok{data=}\NormalTok{train\_normalized)}

\CommentTok{\# Fórmulas com maior grau podem ser criadas da mesma maneira...}
\end{Highlighting}
\end{Shaded}

\subsubsection{Módulo 6.2: Função para criar fórmulas
automaticamente}\label{muxf3dulo-6.2-funuxe7uxe3o-para-criar-fuxf3rmulas-automaticamente}

\begin{Shaded}
\begin{Highlighting}[]
\NormalTok{getHypothesis }\OtherTok{\textless{}{-}} \ControlFlowTok{function}\NormalTok{(real\_feature\_names, degree)\{}
\NormalTok{  hypothesis\_string }\OtherTok{\textless{}{-}} \StringTok{"hypothesis \textless{}{-} formula(median\_house\_value \textasciitilde{} "}
  
  \ControlFlowTok{for}\NormalTok{(d }\ControlFlowTok{in} \DecValTok{1}\SpecialCharTok{:}\NormalTok{degree)\{}
    \ControlFlowTok{for}\NormalTok{(i }\ControlFlowTok{in} \DecValTok{1}\SpecialCharTok{:}\FunctionTok{length}\NormalTok{(real\_feature\_names))\{}
\NormalTok{      hypothesis\_string }\OtherTok{\textless{}{-}} \FunctionTok{paste}\NormalTok{(hypothesis\_string, }
                                 \StringTok{"I("}\NormalTok{, real\_feature\_names[i], }\StringTok{"\^{}"}\NormalTok{, d, }\StringTok{") + "}\NormalTok{,}
                                 \AttributeTok{sep =} \StringTok{""}\NormalTok{)}
\NormalTok{    \}}
\NormalTok{  \}}
  
\NormalTok{  hypothesis\_string }\OtherTok{\textless{}{-}} \FunctionTok{substr}\NormalTok{(hypothesis\_string, }\DecValTok{1}\NormalTok{, }\FunctionTok{nchar}\NormalTok{(hypothesis\_string)}\SpecialCharTok{{-}}\DecValTok{3}\NormalTok{)}
\NormalTok{  hypothesis\_string }\OtherTok{\textless{}{-}} \FunctionTok{paste}\NormalTok{(hypothesis\_string, }\StringTok{")"}\NormalTok{)}
\NormalTok{  hypothesis }\OtherTok{\textless{}{-}} \FunctionTok{eval}\NormalTok{(}\FunctionTok{parse}\NormalTok{(}\AttributeTok{text=}\NormalTok{hypothesis\_string))}
  \FunctionTok{return}\NormalTok{(hypothesis)}
\NormalTok{\}}
\end{Highlighting}
\end{Shaded}

\begin{Shaded}
\begin{Highlighting}[]
\NormalTok{f01 }\OtherTok{\textless{}{-}} \FunctionTok{getHypothesis}\NormalTok{(}\FunctionTok{colnames}\NormalTok{(train\_normalized)[}\DecValTok{1}\SpecialCharTok{:}\DecValTok{8}\NormalTok{], }\AttributeTok{degree=}\DecValTok{1}\NormalTok{)}
\NormalTok{f02 }\OtherTok{\textless{}{-}} \FunctionTok{getHypothesis}\NormalTok{(}\FunctionTok{colnames}\NormalTok{(train\_normalized)[}\DecValTok{1}\SpecialCharTok{:}\DecValTok{8}\NormalTok{], }\AttributeTok{degree=}\DecValTok{2}\NormalTok{)}
\NormalTok{f03 }\OtherTok{\textless{}{-}} \FunctionTok{getHypothesis}\NormalTok{(}\FunctionTok{colnames}\NormalTok{(train\_normalized)[}\DecValTok{1}\SpecialCharTok{:}\DecValTok{8}\NormalTok{], }\AttributeTok{degree=}\DecValTok{3}\NormalTok{)}
\NormalTok{f04 }\OtherTok{\textless{}{-}} \FunctionTok{getHypothesis}\NormalTok{(}\FunctionTok{colnames}\NormalTok{(train\_normalized)[}\DecValTok{1}\SpecialCharTok{:}\DecValTok{8}\NormalTok{], }\AttributeTok{degree=}\DecValTok{4}\NormalTok{)}
\NormalTok{f05 }\OtherTok{\textless{}{-}} \FunctionTok{getHypothesis}\NormalTok{(}\FunctionTok{colnames}\NormalTok{(train\_normalized)[}\DecValTok{1}\SpecialCharTok{:}\DecValTok{8}\NormalTok{], }\AttributeTok{degree=}\DecValTok{5}\NormalTok{)}
\NormalTok{f06 }\OtherTok{\textless{}{-}} \FunctionTok{getHypothesis}\NormalTok{(}\FunctionTok{colnames}\NormalTok{(train\_normalized)[}\DecValTok{1}\SpecialCharTok{:}\DecValTok{8}\NormalTok{], }\AttributeTok{degree=}\DecValTok{6}\NormalTok{)}
\NormalTok{f07 }\OtherTok{\textless{}{-}} \FunctionTok{getHypothesis}\NormalTok{(}\FunctionTok{colnames}\NormalTok{(train\_normalized)[}\DecValTok{1}\SpecialCharTok{:}\DecValTok{8}\NormalTok{], }\AttributeTok{degree=}\DecValTok{7}\NormalTok{)}
\NormalTok{f08 }\OtherTok{\textless{}{-}} \FunctionTok{getHypothesis}\NormalTok{(}\FunctionTok{colnames}\NormalTok{(train\_normalized)[}\DecValTok{1}\SpecialCharTok{:}\DecValTok{8}\NormalTok{], }\AttributeTok{degree=}\DecValTok{8}\NormalTok{)}
\NormalTok{f09 }\OtherTok{\textless{}{-}} \FunctionTok{getHypothesis}\NormalTok{(}\FunctionTok{colnames}\NormalTok{(train\_normalized)[}\DecValTok{1}\SpecialCharTok{:}\DecValTok{8}\NormalTok{], }\AttributeTok{degree=}\DecValTok{9}\NormalTok{)}
\NormalTok{f10 }\OtherTok{\textless{}{-}} \FunctionTok{getHypothesis}\NormalTok{(}\FunctionTok{colnames}\NormalTok{(train\_normalized)[}\DecValTok{1}\SpecialCharTok{:}\DecValTok{8}\NormalTok{], }\AttributeTok{degree=}\DecValTok{10}\NormalTok{)}

\NormalTok{f10}
\end{Highlighting}
\end{Shaded}

\begin{verbatim}
## median_house_value ~ I(longitude^1) + I(latitude^1) + I(housing_median_age^1) + 
##     I(total_rooms^1) + I(total_bedrooms^1) + I(population^1) + 
##     I(households^1) + I(median_income^1) + I(longitude^2) + I(latitude^2) + 
##     I(housing_median_age^2) + I(total_rooms^2) + I(total_bedrooms^2) + 
##     I(population^2) + I(households^2) + I(median_income^2) + 
##     I(longitude^3) + I(latitude^3) + I(housing_median_age^3) + 
##     I(total_rooms^3) + I(total_bedrooms^3) + I(population^3) + 
##     I(households^3) + I(median_income^3) + I(longitude^4) + I(latitude^4) + 
##     I(housing_median_age^4) + I(total_rooms^4) + I(total_bedrooms^4) + 
##     I(population^4) + I(households^4) + I(median_income^4) + 
##     I(longitude^5) + I(latitude^5) + I(housing_median_age^5) + 
##     I(total_rooms^5) + I(total_bedrooms^5) + I(population^5) + 
##     I(households^5) + I(median_income^5) + I(longitude^6) + I(latitude^6) + 
##     I(housing_median_age^6) + I(total_rooms^6) + I(total_bedrooms^6) + 
##     I(population^6) + I(households^6) + I(median_income^6) + 
##     I(longitude^7) + I(latitude^7) + I(housing_median_age^7) + 
##     I(total_rooms^7) + I(total_bedrooms^7) + I(population^7) + 
##     I(households^7) + I(median_income^7) + I(longitude^8) + I(latitude^8) + 
##     I(housing_median_age^8) + I(total_rooms^8) + I(total_bedrooms^8) + 
##     I(population^8) + I(households^8) + I(median_income^8) + 
##     I(longitude^9) + I(latitude^9) + I(housing_median_age^9) + 
##     I(total_rooms^9) + I(total_bedrooms^9) + I(population^9) + 
##     I(households^9) + I(median_income^9) + I(longitude^10) + 
##     I(latitude^10) + I(housing_median_age^10) + I(total_rooms^10) + 
##     I(total_bedrooms^10) + I(population^10) + I(households^10) + 
##     I(median_income^10)
## <environment: 0x0000018ce426c268>
\end{verbatim}

\subsubsection{Módulo 6.3: Treinamento}\label{muxf3dulo-6.3-treinamento}

\begin{Shaded}
\begin{Highlighting}[]
\NormalTok{formulas }\OtherTok{\textless{}{-}} \FunctionTok{list}\NormalTok{(f01, f02, f03, f04, f05, f06, f07, f08, f09, f10)}
\NormalTok{MaePerDegree }\OtherTok{\textless{}{-}} \FunctionTok{data.frame}\NormalTok{(}\AttributeTok{Degree   =} \FunctionTok{numeric}\NormalTok{(}\FunctionTok{length}\NormalTok{(formulas)), }
                           \AttributeTok{TrainMAE =} \FunctionTok{numeric}\NormalTok{(}\FunctionTok{length}\NormalTok{(formulas)),}
                           \AttributeTok{ValMAE   =} \FunctionTok{numeric}\NormalTok{(}\FunctionTok{length}\NormalTok{(formulas)))}

\NormalTok{i }\OtherTok{\textless{}{-}} \DecValTok{1}
\ControlFlowTok{for}\NormalTok{(f }\ControlFlowTok{in}\NormalTok{ formulas)\{}
\NormalTok{  model }\OtherTok{\textless{}{-}} \FunctionTok{lm}\NormalTok{(}\AttributeTok{formula=}\NormalTok{f, }\AttributeTok{data=}\NormalTok{train\_normalized)}
  
\NormalTok{  train\_pred }\OtherTok{\textless{}{-}} \FunctionTok{predict}\NormalTok{(model, train\_normalized)}
\NormalTok{  valid\_pred }\OtherTok{\textless{}{-}} \FunctionTok{predict}\NormalTok{(model, valid\_normalized)}
  
\NormalTok{  mae\_train }\OtherTok{\textless{}{-}} \FunctionTok{MAE}\NormalTok{(train\_pred, train\_normalized}\SpecialCharTok{$}\NormalTok{median\_house\_value)}
\NormalTok{  mae\_val   }\OtherTok{\textless{}{-}} \FunctionTok{MAE}\NormalTok{(valid\_pred, valid\_normalized}\SpecialCharTok{$}\NormalTok{median\_house\_value)}
 
\NormalTok{  MaePerDegree[i,] }\OtherTok{=} \FunctionTok{c}\NormalTok{(i, mae\_train, mae\_val)}
\NormalTok{  i }\OtherTok{\textless{}{-}}\NormalTok{ i }\SpecialCharTok{+} \DecValTok{1}
\NormalTok{\}}
\end{Highlighting}
\end{Shaded}

\subsubsection{Módulo 6.4: Curva de Viés e
Variância}\label{muxf3dulo-6.4-curva-de-viuxe9s-e-variuxe2ncia}

\begin{Shaded}
\begin{Highlighting}[]
\NormalTok{MaePerDegree}
\end{Highlighting}
\end{Shaded}

\begin{verbatim}
##    Degree TrainMAE ValMAE
## 1       1    50609  51594
## 2       2    49265  50234
## 3       3    46721  47583
## 4       4    45980  46871
## 5       5    45225  46102
## 6       6    44995  45905
## 7       7    44696  45649
## 8       8    44475  47372
## 9       9    44453  45441
## 10     10    44038  55193
\end{verbatim}

\begin{Shaded}
\begin{Highlighting}[]
\NormalTok{MaePerDegreeMelt }\OtherTok{\textless{}{-}} \FunctionTok{melt}\NormalTok{(MaePerDegree, }\AttributeTok{id=}\StringTok{"Degree"}\NormalTok{)}

\NormalTok{p }\OtherTok{\textless{}{-}} \FunctionTok{ggplot}\NormalTok{(}\AttributeTok{data=}\NormalTok{MaePerDegreeMelt, }\FunctionTok{aes}\NormalTok{(}\AttributeTok{x=}\NormalTok{Degree, }\AttributeTok{y=}\NormalTok{value, }\AttributeTok{colour=}\NormalTok{variable)) }\SpecialCharTok{+} \FunctionTok{geom\_line}\NormalTok{() }\SpecialCharTok{+} \FunctionTok{geom\_point}\NormalTok{()}

\NormalTok{p }\OtherTok{\textless{}{-}}\NormalTok{ p }\SpecialCharTok{+} \FunctionTok{ggtitle}\NormalTok{(}\StringTok{"Curva vies/variancia"}\NormalTok{) }\SpecialCharTok{+} \FunctionTok{ylab}\NormalTok{(}\StringTok{"MAE"}\NormalTok{) }\SpecialCharTok{+} \FunctionTok{scale\_x\_discrete}\NormalTok{(}\AttributeTok{name =}\StringTok{"Degree"}\NormalTok{, }\AttributeTok{limits=}\FunctionTok{as.character}\NormalTok{(}\DecValTok{1}\SpecialCharTok{:}\FunctionTok{length}\NormalTok{(formulas)))}

\NormalTok{p }\SpecialCharTok{+} \FunctionTok{theme}\NormalTok{(}\AttributeTok{legend.position =} \FunctionTok{c}\NormalTok{(}\FloatTok{0.7}\NormalTok{, }\FloatTok{0.85}\NormalTok{), }\AttributeTok{legend.title =} \FunctionTok{element\_blank}\NormalTok{())}
\end{Highlighting}
\end{Shaded}

\pandocbounded{\includegraphics[keepaspectratio]{Ex01_files/figure-latex/unnamed-chunk-32-1.pdf}}

\begin{center}\rule{0.5\linewidth}{0.5pt}\end{center}

\subsection{Módulo 7: Inserção de Features
Categóricas}\label{muxf3dulo-7-inseruxe7uxe3o-de-features-categuxf3ricas}

Neste Módulo, vamos expandir o modelo para incluir as \textbf{features
categóricas}.

\subsubsection{Módulo 7.1: Modelo Baseline com Features
Categóricas}\label{muxf3dulo-7.1-modelo-baseline-com-features-categuxf3ricas}

\begin{Shaded}
\begin{Highlighting}[]
\NormalTok{cat\_model }\OtherTok{\textless{}{-}} \FunctionTok{lm}\NormalTok{(}\AttributeTok{formula =}\NormalTok{ median\_house\_value }\SpecialCharTok{\textasciitilde{}}\NormalTok{ longitude }\SpecialCharTok{+}\NormalTok{ latitude }
                \SpecialCharTok{+}\NormalTok{ housing\_median\_age }\SpecialCharTok{+}\NormalTok{ total\_rooms }\SpecialCharTok{+}\NormalTok{ total\_bedrooms}
                \SpecialCharTok{+}\NormalTok{ population }\SpecialCharTok{+}\NormalTok{ households }\SpecialCharTok{+}\NormalTok{ median\_income }\SpecialCharTok{+}\NormalTok{ less1Hocean }\SpecialCharTok{+}\NormalTok{ inland}
                \SpecialCharTok{+}\NormalTok{ island }\SpecialCharTok{+}\NormalTok{ nearbay }\SpecialCharTok{+}\NormalTok{ nearocean, }\AttributeTok{data =}\NormalTok{ train\_normalized)}

\FunctionTok{summary}\NormalTok{(cat\_model)}
\end{Highlighting}
\end{Shaded}

\begin{verbatim}
## 
## Call:
## lm(formula = median_house_value ~ longitude + latitude + housing_median_age + 
##     total_rooms + total_bedrooms + population + households + 
##     median_income + less1Hocean + inland + island + nearbay + 
##     nearocean, data = train_normalized)
## 
## Residuals:
##     Min      1Q  Median      3Q     Max 
## -381603  -42799  -10205   29004  687391 
## 
## Coefficients: (1 not defined because of singularities)
##                    Estimate Std. Error t value Pr(>|t|)    
## (Intercept)          249765      10251   24.37  < 2e-16 ***
## longitude           -261908      13164  -19.90  < 2e-16 ***
## latitude            -233440      12122  -19.26  < 2e-16 ***
## housing_median_age    57712       2874   20.08  < 2e-16 ***
## total_rooms         -317994      38562   -8.25  < 2e-16 ***
## total_bedrooms       708333      55751   12.71  < 2e-16 ***
## population         -1201354      46304  -25.95  < 2e-16 ***
## households           239104      55927    4.28  1.9e-05 ***
## median_income        581228       6320   91.97  < 2e-16 ***
## less1Hocean           -4931       1997   -2.47    0.014 *  
## inland               -43068       2872  -15.00  < 2e-16 ***
## island               155555      39417    3.95  8.0e-05 ***
## nearbay               -7316       2758   -2.65    0.008 ** 
## nearocean                NA         NA      NA       NA    
## ---
## Signif. codes:  0 '***' 0.001 '**' 0.01 '*' 0.05 '.' 0.1 ' ' 1
## 
## Residual standard error: 68200 on 12255 degrees of freedom
## Multiple R-squared:  0.649,  Adjusted R-squared:  0.649 
## F-statistic: 1.89e+03 on 12 and 12255 DF,  p-value: <2e-16
\end{verbatim}

\subsubsection{Módulo 7.2: Treinamento e
Validação}\label{muxf3dulo-7.2-treinamento-e-validauxe7uxe3o}

\begin{Shaded}
\begin{Highlighting}[]
\CommentTok{\# Predição nos dados de treino e validação}
\NormalTok{train\_pred }\OtherTok{\textless{}{-}} \FunctionTok{predict}\NormalTok{(cat\_model, train\_normalized)}
\NormalTok{valid\_pred }\OtherTok{\textless{}{-}} \FunctionTok{predict}\NormalTok{(cat\_model, valid\_normalized)}

\CommentTok{\# Cálculo das métricas de erro}
\NormalTok{mae\_train }\OtherTok{\textless{}{-}} \FunctionTok{MAE}\NormalTok{(train\_pred, train\_normalized}\SpecialCharTok{$}\NormalTok{median\_house\_value)}
\NormalTok{mse\_train }\OtherTok{\textless{}{-}} \FunctionTok{MSE}\NormalTok{(train\_pred, train\_normalized}\SpecialCharTok{$}\NormalTok{median\_house\_value)}
\NormalTok{r2\_train  }\OtherTok{\textless{}{-}} \FunctionTok{R2}\NormalTok{(train\_pred,  train\_normalized}\SpecialCharTok{$}\NormalTok{median\_house\_value)}

\NormalTok{mae\_valid }\OtherTok{\textless{}{-}} \FunctionTok{MAE}\NormalTok{(valid\_pred, valid\_normalized}\SpecialCharTok{$}\NormalTok{median\_house\_value)}
\NormalTok{mse\_valid }\OtherTok{\textless{}{-}} \FunctionTok{MSE}\NormalTok{(valid\_pred, valid\_normalized}\SpecialCharTok{$}\NormalTok{median\_house\_value)}
\NormalTok{r2\_valid  }\OtherTok{\textless{}{-}} \FunctionTok{R2}\NormalTok{(valid\_pred,  valid\_normalized}\SpecialCharTok{$}\NormalTok{median\_house\_value)}

\NormalTok{results\_cat }\OtherTok{\textless{}{-}} \FunctionTok{data.frame}\NormalTok{(}
  \AttributeTok{Metric =} \FunctionTok{c}\NormalTok{(}\StringTok{"MAE"}\NormalTok{, }\StringTok{"MSE"}\NormalTok{, }\StringTok{"R2"}\NormalTok{),}
  \AttributeTok{Train  =} \FunctionTok{c}\NormalTok{(mae\_train, mse\_train, r2\_train),}
  \AttributeTok{Valid  =} \FunctionTok{c}\NormalTok{(mae\_valid, mse\_valid, r2\_valid)}
\NormalTok{)}

\NormalTok{results\_cat}\SpecialCharTok{$}\NormalTok{Train }\OtherTok{\textless{}{-}} \FunctionTok{as.numeric}\NormalTok{(results\_cat}\SpecialCharTok{$}\NormalTok{Train)}
\NormalTok{results\_cat}\SpecialCharTok{$}\NormalTok{Valid }\OtherTok{\textless{}{-}} \FunctionTok{as.numeric}\NormalTok{(results\_cat}\SpecialCharTok{$}\NormalTok{Valid)}
\NormalTok{results\_cat}\SpecialCharTok{$}\NormalTok{Train }\OtherTok{\textless{}{-}} \FunctionTok{format}\NormalTok{(results\_cat}\SpecialCharTok{$}\NormalTok{Train, }\AttributeTok{scientific =} \ConstantTok{FALSE}\NormalTok{)}
\NormalTok{results\_cat}\SpecialCharTok{$}\NormalTok{Valid }\OtherTok{\textless{}{-}} \FunctionTok{format}\NormalTok{(results\_cat}\SpecialCharTok{$}\NormalTok{Valid, }\AttributeTok{scientific =} \ConstantTok{FALSE}\NormalTok{)}

\NormalTok{results\_cat}
\end{Highlighting}
\end{Shaded}

\begin{verbatim}
##   Metric          Train          Valid
## 1    MAE      49604.590      50473.328
## 2    MSE 4642657714.665 4916793798.668
## 3     R2          0.649          0.634
\end{verbatim}

\subsubsection{Módulo 7.3: Combinação de
Features}\label{muxf3dulo-7.3-combinauxe7uxe3o-de-features}

\begin{Shaded}
\begin{Highlighting}[]
\NormalTok{f01\_cat }\OtherTok{\textless{}{-}} \FunctionTok{formula}\NormalTok{(median\_house\_value }\SpecialCharTok{\textasciitilde{}}\NormalTok{ longitude }\SpecialCharTok{+}\NormalTok{ latitude }\SpecialCharTok{+}\NormalTok{ housing\_median\_age}
                   \SpecialCharTok{+}\NormalTok{ total\_rooms }\SpecialCharTok{+}\NormalTok{ total\_bedrooms }\SpecialCharTok{+}\NormalTok{ households }\SpecialCharTok{+}\NormalTok{ median\_income}
                   \SpecialCharTok{+}\NormalTok{ population }\SpecialCharTok{+}\NormalTok{ (longitude }\SpecialCharTok{+}\NormalTok{ latitude }\SpecialCharTok{+}\NormalTok{ housing\_median\_age}
                   \SpecialCharTok{+}\NormalTok{ total\_rooms }\SpecialCharTok{+}\NormalTok{ total\_bedrooms }\SpecialCharTok{+}\NormalTok{ households }\SpecialCharTok{+}\NormalTok{ median\_income}
                   \SpecialCharTok{+}\NormalTok{ population)}\SpecialCharTok{\^{}}\DecValTok{2} \SpecialCharTok{+}\NormalTok{ less1Hocean }\SpecialCharTok{+}\NormalTok{ inland }\SpecialCharTok{+}\NormalTok{ island }\SpecialCharTok{+}\NormalTok{ nearbay}
                   \SpecialCharTok{+}\NormalTok{ nearocean)}

\NormalTok{f02\_cat }\OtherTok{\textless{}{-}} \FunctionTok{formula}\NormalTok{(median\_house\_value }\SpecialCharTok{\textasciitilde{}}\NormalTok{ longitude }\SpecialCharTok{+}\NormalTok{ latitude }\SpecialCharTok{+}\NormalTok{ housing\_median\_age}
                   \SpecialCharTok{+}\NormalTok{ total\_rooms }\SpecialCharTok{+}\NormalTok{ total\_bedrooms }\SpecialCharTok{+}\NormalTok{ households }\SpecialCharTok{+}\NormalTok{ median\_income}
                   \SpecialCharTok{+}\NormalTok{ population }\SpecialCharTok{+}\NormalTok{ (longitude }\SpecialCharTok{+}\NormalTok{ latitude }\SpecialCharTok{+}\NormalTok{ housing\_median\_age}
                   \SpecialCharTok{+}\NormalTok{ total\_rooms }\SpecialCharTok{+}\NormalTok{ total\_bedrooms }\SpecialCharTok{+}\NormalTok{ households }\SpecialCharTok{+}\NormalTok{ median\_income}
                   \SpecialCharTok{+}\NormalTok{ population)}\SpecialCharTok{\^{}}\DecValTok{3} \SpecialCharTok{+}\NormalTok{less1Hocean }\SpecialCharTok{+}\NormalTok{ inland }\SpecialCharTok{+}\NormalTok{ island }\SpecialCharTok{+}\NormalTok{ nearbay}
                   \SpecialCharTok{+}\NormalTok{ nearocean)}

\NormalTok{f03\_cat }\OtherTok{\textless{}{-}} \FunctionTok{formula}\NormalTok{(median\_house\_value }\SpecialCharTok{\textasciitilde{}}\NormalTok{ longitude }\SpecialCharTok{+}\NormalTok{ latitude }\SpecialCharTok{+}\NormalTok{ housing\_median\_age}
                   \SpecialCharTok{+}\NormalTok{ total\_rooms }\SpecialCharTok{+}\NormalTok{ total\_bedrooms }\SpecialCharTok{+}\NormalTok{ households }\SpecialCharTok{+}\NormalTok{ median\_income}
                   \SpecialCharTok{+}\NormalTok{ (longitude }\SpecialCharTok{+}\NormalTok{ latitude }\SpecialCharTok{+}\NormalTok{ housing\_median\_age }\SpecialCharTok{+}\NormalTok{ total\_rooms}
                   \SpecialCharTok{+}\NormalTok{ total\_bedrooms }\SpecialCharTok{+}\NormalTok{ households }\SpecialCharTok{+}\NormalTok{ median\_income }\SpecialCharTok{+}\NormalTok{ population)}\SpecialCharTok{\^{}}\DecValTok{4}
                   \SpecialCharTok{+}\NormalTok{ less1Hocean }\SpecialCharTok{+}\NormalTok{ inland }\SpecialCharTok{+}\NormalTok{ island }\SpecialCharTok{+}\NormalTok{ nearbay }\SpecialCharTok{+}\NormalTok{ nearocean)}
\end{Highlighting}
\end{Shaded}

\begin{Shaded}
\begin{Highlighting}[]
\NormalTok{formulas }\OtherTok{\textless{}{-}} \FunctionTok{c}\NormalTok{(f01\_cat, f02\_cat, f03\_cat)}
\NormalTok{MaePerCombinationCat }\OtherTok{\textless{}{-}} \FunctionTok{data.frame}\NormalTok{(}\AttributeTok{Formula  =} \FunctionTok{numeric}\NormalTok{(}\FunctionTok{length}\NormalTok{(formulas)), }
                                \AttributeTok{TrainCATMAE =} \FunctionTok{numeric}\NormalTok{(}\FunctionTok{length}\NormalTok{(formulas)),}
                                \AttributeTok{ValCATMAE   =} \FunctionTok{numeric}\NormalTok{(}\FunctionTok{length}\NormalTok{(formulas)))}

\NormalTok{i }\OtherTok{\textless{}{-}} \DecValTok{1}
\ControlFlowTok{for}\NormalTok{(f }\ControlFlowTok{in}\NormalTok{ formulas)\{}
  
\NormalTok{  model }\OtherTok{\textless{}{-}} \FunctionTok{lm}\NormalTok{(}\AttributeTok{formula=}\NormalTok{f, }\AttributeTok{data=}\NormalTok{train\_normalized)}
  
\NormalTok{  train\_pred }\OtherTok{\textless{}{-}} \FunctionTok{predict}\NormalTok{(model, train\_normalized)}
\NormalTok{  valid\_pred }\OtherTok{\textless{}{-}} \FunctionTok{predict}\NormalTok{(model, valid\_normalized)}
  
\NormalTok{  mae\_train }\OtherTok{\textless{}{-}} \FunctionTok{MAE}\NormalTok{(train\_pred, train\_normalized}\SpecialCharTok{$}\NormalTok{median\_house\_value)}
\NormalTok{  mae\_val   }\OtherTok{\textless{}{-}} \FunctionTok{MAE}\NormalTok{(valid\_pred, valid\_normalized}\SpecialCharTok{$}\NormalTok{median\_house\_value)}
 
\NormalTok{  MaePerCombinationCat[i,] }\OtherTok{=} \FunctionTok{c}\NormalTok{(i, mae\_train, mae\_val)}
\NormalTok{  i }\OtherTok{\textless{}{-}}\NormalTok{ i }\SpecialCharTok{+} \DecValTok{1}
\NormalTok{\}}
\end{Highlighting}
\end{Shaded}

\subsection{Módulo 7.4: Curva de Viés e
Variância}\label{muxf3dulo-7.4-curva-de-viuxe9s-e-variuxe2ncia}

\begin{Shaded}
\begin{Highlighting}[]
\NormalTok{MaePerCombinationCat}
\end{Highlighting}
\end{Shaded}

\begin{verbatim}
##   Formula TrainCATMAE ValCATMAE
## 1       1       45650     46842
## 2       2       43926     45225
## 3       3       43043     44916
\end{verbatim}

\begin{Shaded}
\begin{Highlighting}[]
\NormalTok{MaePerCombinationCatMelt }\OtherTok{\textless{}{-}} \FunctionTok{melt}\NormalTok{(MaePerCombinationCat, }\AttributeTok{id =} \StringTok{"Formula"}\NormalTok{)}

\CommentTok{\# Criando o gráfico da curva de viés e variância}
\NormalTok{p }\OtherTok{\textless{}{-}} \FunctionTok{ggplot}\NormalTok{(}\AttributeTok{data =}\NormalTok{ MaePerCombinationCatMelt, }\FunctionTok{aes}\NormalTok{(}\AttributeTok{x =}\NormalTok{ Formula, }\AttributeTok{y =}\NormalTok{ value, }\AttributeTok{colour =}\NormalTok{ variable)) }\SpecialCharTok{+}
  \FunctionTok{geom\_line}\NormalTok{(}\AttributeTok{linetype =} \StringTok{"dashed"}\NormalTok{) }\SpecialCharTok{+} 
  \FunctionTok{geom\_point}\NormalTok{()}


\NormalTok{p }\OtherTok{\textless{}{-}}\NormalTok{ p }\SpecialCharTok{+} \FunctionTok{ggtitle}\NormalTok{(}\StringTok{"Curva Viés/Variância"}\NormalTok{) }\SpecialCharTok{+}
  \FunctionTok{ylab}\NormalTok{(}\StringTok{"MAE"}\NormalTok{) }\SpecialCharTok{+}
  \FunctionTok{scale\_x\_discrete}\NormalTok{(}\AttributeTok{name =} \StringTok{"Formula"}\NormalTok{, }\AttributeTok{limits =} \FunctionTok{as.character}\NormalTok{(}\DecValTok{1}\SpecialCharTok{:}\FunctionTok{length}\NormalTok{(formulas)))}


\NormalTok{p }\OtherTok{\textless{}{-}}\NormalTok{ p }\SpecialCharTok{+} \FunctionTok{theme}\NormalTok{(}\AttributeTok{legend.position =} \FunctionTok{c}\NormalTok{(}\FloatTok{0.7}\NormalTok{, }\FloatTok{0.85}\NormalTok{), }\AttributeTok{legend.title =} \FunctionTok{element\_blank}\NormalTok{())}
\NormalTok{p }\OtherTok{\textless{}{-}}\NormalTok{ p }\SpecialCharTok{+} \FunctionTok{geom\_line}\NormalTok{(}\AttributeTok{data =}\NormalTok{ MaePerCombinationMelt) }\SpecialCharTok{+} 
  \FunctionTok{geom\_point}\NormalTok{(}\AttributeTok{data =}\NormalTok{ MaePerCombinationMelt)}
\NormalTok{p }\SpecialCharTok{+} \FunctionTok{scale\_color\_manual}\NormalTok{(}\AttributeTok{values =} \FunctionTok{c}\NormalTok{(}\StringTok{"TrainMAE"} \OtherTok{=} \StringTok{"red"}\NormalTok{, }\StringTok{"TrainCATMAE"} \OtherTok{=} \StringTok{"orange"}\NormalTok{,}
                                  \StringTok{"ValMAE"} \OtherTok{=} \StringTok{"blue"}\NormalTok{, }\StringTok{"ValCATMAE"} \OtherTok{=} \StringTok{"deepskyblue3"}\NormalTok{))}
\end{Highlighting}
\end{Shaded}

\pandocbounded{\includegraphics[keepaspectratio]{Ex01_files/figure-latex/unnamed-chunk-37-1.pdf}}

\begin{center}\rule{0.5\linewidth}{0.5pt}\end{center}

\subsection{Módulo 8: Conjunto de
Teste}\label{muxf3dulo-8-conjunto-de-teste}

Após o treinamento e a validação de diferentes modelos, é necessário
selecionar o melhor modelo com base no conjunto de validação para
avaliá-lo no conjunto de teste.

\textbf{Importante:} O conjunto de teste deve ser utilizado
\textbf{apenas uma vez}. O desempenho do modelo no conjunto de teste
reflete sua capacidade de generalização para o mundo real.

\subsubsection{Módulo 8.1: Avaliando o Desempenho dos
Modelos}\label{muxf3dulo-8.1-avaliando-o-desempenho-dos-modelos}

\begin{Shaded}
\begin{Highlighting}[]
\FunctionTok{min}\NormalTok{(MaePerCombination}\SpecialCharTok{$}\NormalTok{ValMAE)        }\CommentTok{\# Modelo baseado em combinações}
\end{Highlighting}
\end{Shaded}

\begin{verbatim}
## [1] 45618
\end{verbatim}

\begin{Shaded}
\begin{Highlighting}[]
\FunctionTok{min}\NormalTok{(MaePerCombinationAI}\SpecialCharTok{$}\NormalTok{ValMAE) }\CommentTok{\# Modelo com ChatGPT}
\end{Highlighting}
\end{Shaded}

\begin{verbatim}
## [1] 51253
\end{verbatim}

\begin{Shaded}
\begin{Highlighting}[]
\FunctionTok{min}\NormalTok{(MaePerDegree}\SpecialCharTok{$}\NormalTok{ValMAE)             }\CommentTok{\# Modelo baseado em graus}
\end{Highlighting}
\end{Shaded}

\begin{verbatim}
## [1] 45441
\end{verbatim}

\begin{Shaded}
\begin{Highlighting}[]
\FunctionTok{min}\NormalTok{(MaePerCombinationCat}\SpecialCharTok{$}\NormalTok{ValCATMAE)  }\CommentTok{\# Modelo baseado em características categóricas}
\end{Highlighting}
\end{Shaded}

\begin{verbatim}
## [1] 44916
\end{verbatim}

\begin{Shaded}
\begin{Highlighting}[]
\CommentTok{\# Exibindo o MAE do modelo categórico}
\NormalTok{MaePerCombinationCat}\SpecialCharTok{$}\NormalTok{ValCATMAE}
\end{Highlighting}
\end{Shaded}

\begin{verbatim}
## [1] 46842 45225 44916
\end{verbatim}

\subsubsection{Módulo 8.2: Preparando o Conjunto de
Teste}\label{muxf3dulo-8.2-preparando-o-conjunto-de-teste}

\begin{Shaded}
\begin{Highlighting}[]
\NormalTok{test\_data }\OtherTok{\textless{}{-}} \FunctionTok{read.csv}\NormalTok{(}\StringTok{"housePricing\_test\_set.csv"}\NormalTok{, }
                      \AttributeTok{header=}\ConstantTok{TRUE}\NormalTok{, }\AttributeTok{stringsAsFactors=}\ConstantTok{TRUE}\NormalTok{)}

\FunctionTok{any}\NormalTok{(}\FunctionTok{is.na}\NormalTok{(test\_data))}
\end{Highlighting}
\end{Shaded}

\begin{verbatim}
## [1] FALSE
\end{verbatim}

\begin{Shaded}
\begin{Highlighting}[]
\NormalTok{test\_encoded }\OtherTok{\textless{}{-}}\NormalTok{ test\_data}

\NormalTok{categories     }\OtherTok{\textless{}{-}} \FunctionTok{c}\NormalTok{(}\StringTok{"\textless{}1H OCEAN"}\NormalTok{, }\StringTok{"INLAND"}\NormalTok{, }\StringTok{"ISLAND"}\NormalTok{, }\StringTok{"NEAR BAY"}\NormalTok{, }\StringTok{"NEAR OCEAN"}\NormalTok{)}
\NormalTok{new\_categories }\OtherTok{\textless{}{-}} \FunctionTok{c}\NormalTok{(}\StringTok{\textquotesingle{}less1Hocean\textquotesingle{}}\NormalTok{, }\StringTok{\textquotesingle{}inland\textquotesingle{}}\NormalTok{, }\StringTok{\textquotesingle{}island\textquotesingle{}}\NormalTok{, }\StringTok{\textquotesingle{}nearbay\textquotesingle{}}\NormalTok{, }\StringTok{\textquotesingle{}nearocean\textquotesingle{}}\NormalTok{)}

\ControlFlowTok{for}\NormalTok{ (i }\ControlFlowTok{in} \FunctionTok{seq\_along}\NormalTok{(categories)) \{}
\NormalTok{  cat     }\OtherTok{\textless{}{-}}\NormalTok{ categories[i]}
\NormalTok{  new\_cat }\OtherTok{\textless{}{-}}\NormalTok{ new\_categories[i]}
  
\NormalTok{  test\_encoded[[new\_cat]] }\OtherTok{\textless{}{-}} \FunctionTok{as.numeric}\NormalTok{(test\_encoded}\SpecialCharTok{$}\NormalTok{ocean\_proximity }\SpecialCharTok{==}\NormalTok{ cat)}
\NormalTok{\}}

\NormalTok{test\_encoded}\SpecialCharTok{$}\NormalTok{ocean\_proximity }\OtherTok{\textless{}{-}} \ConstantTok{NULL}
\end{Highlighting}
\end{Shaded}

\begin{Shaded}
\begin{Highlighting}[]
\NormalTok{test\_normalized }\OtherTok{\textless{}{-}}\NormalTok{ test\_encoded}

\NormalTok{test\_normalized[,}\DecValTok{1}\SpecialCharTok{:}\DecValTok{8}\NormalTok{]  }\OtherTok{\textless{}{-}} \FunctionTok{sweep}\NormalTok{(test\_normalized[,}\DecValTok{1}\SpecialCharTok{:}\DecValTok{8}\NormalTok{], }\DecValTok{2}\NormalTok{, min\_features, }\StringTok{"{-}"}\NormalTok{)}
\NormalTok{test\_normalized[,}\DecValTok{1}\SpecialCharTok{:}\DecValTok{8}\NormalTok{]  }\OtherTok{\textless{}{-}} \FunctionTok{sweep}\NormalTok{(test\_normalized[,}\DecValTok{1}\SpecialCharTok{:}\DecValTok{8}\NormalTok{], }\DecValTok{2}\NormalTok{, diff, }\StringTok{"/"}\NormalTok{)}

\NormalTok{test\_normalized[, }\DecValTok{1}\SpecialCharTok{:}\DecValTok{8}\NormalTok{] }\OtherTok{\textless{}{-}} \FunctionTok{as.data.frame}\NormalTok{(}\FunctionTok{lapply}\NormalTok{(test\_normalized[, }\DecValTok{1}\SpecialCharTok{:}\DecValTok{8}\NormalTok{], }\ControlFlowTok{function}\NormalTok{(x) }\FunctionTok{pmax}\NormalTok{(}\DecValTok{0}\NormalTok{, }\FunctionTok{pmin}\NormalTok{(}\DecValTok{1}\NormalTok{, x))))}
\end{Highlighting}
\end{Shaded}

\subsubsection{Módulo 8.3: Avaliando o
modelo}\label{muxf3dulo-8.3-avaliando-o-modelo}

\begin{Shaded}
\begin{Highlighting}[]
\NormalTok{best\_model }\OtherTok{\textless{}{-}} \FunctionTok{lm}\NormalTok{(}\AttributeTok{formula=}\NormalTok{median\_house\_value }\SpecialCharTok{\textasciitilde{}}\NormalTok{ longitude }\SpecialCharTok{+}\NormalTok{ latitude}
                 \SpecialCharTok{+}\NormalTok{ housing\_median\_age }\SpecialCharTok{+}\NormalTok{ total\_rooms }\SpecialCharTok{+}\NormalTok{ total\_bedrooms}
                 \SpecialCharTok{+}\NormalTok{ households }\SpecialCharTok{+}\NormalTok{ median\_income }\SpecialCharTok{+}\NormalTok{ (longitude }\SpecialCharTok{+}\NormalTok{ latitude}
                 \SpecialCharTok{+}\NormalTok{ housing\_median\_age }\SpecialCharTok{+}\NormalTok{ total\_rooms }\SpecialCharTok{+}\NormalTok{ total\_bedrooms }
                 \SpecialCharTok{+}\NormalTok{ households }\SpecialCharTok{+}\NormalTok{ median\_income }\SpecialCharTok{+}\NormalTok{ population)}\SpecialCharTok{\^{}}\DecValTok{4} \SpecialCharTok{+}\NormalTok{ less1Hocean}
                 \SpecialCharTok{+}\NormalTok{ inland }\SpecialCharTok{+}\NormalTok{ island }\SpecialCharTok{+}\NormalTok{ nearbay }\SpecialCharTok{+}\NormalTok{ nearocean, }\AttributeTok{data=}\NormalTok{train\_normalized)}


\NormalTok{test\_pred }\OtherTok{\textless{}{-}} \FunctionTok{predict}\NormalTok{(best\_model, test\_normalized)}

\NormalTok{mae\_test }\OtherTok{\textless{}{-}} \FunctionTok{MAE}\NormalTok{(test\_pred, test\_normalized}\SpecialCharTok{$}\NormalTok{median\_house\_value)}
\NormalTok{mse\_test }\OtherTok{\textless{}{-}} \FunctionTok{MSE}\NormalTok{(test\_pred, test\_normalized}\SpecialCharTok{$}\NormalTok{median\_house\_value)}
\NormalTok{r2\_test  }\OtherTok{\textless{}{-}} \FunctionTok{R2}\NormalTok{(test\_pred,  test\_normalized}\SpecialCharTok{$}\NormalTok{median\_house\_value)}

\NormalTok{results\_cat }\OtherTok{\textless{}{-}} \FunctionTok{data.frame}\NormalTok{(}
  \AttributeTok{Metric =} \FunctionTok{c}\NormalTok{(}\StringTok{"MAE"}\NormalTok{, }\StringTok{"MSE"}\NormalTok{, }\StringTok{"R2"}\NormalTok{),}
  \AttributeTok{Test =} \FunctionTok{c}\NormalTok{(mae\_test, mse\_test, r2\_test)}
\NormalTok{)}

\NormalTok{results\_cat}\SpecialCharTok{$}\NormalTok{Test }\OtherTok{\textless{}{-}} \FunctionTok{as.numeric}\NormalTok{(results\_cat}\SpecialCharTok{$}\NormalTok{Test)}
\NormalTok{results\_cat}\SpecialCharTok{$}\NormalTok{Test }\OtherTok{\textless{}{-}} \FunctionTok{format}\NormalTok{(results\_cat}\SpecialCharTok{$}\NormalTok{Test, }\AttributeTok{scientific =} \ConstantTok{FALSE}\NormalTok{)}

\NormalTok{results\_cat}
\end{Highlighting}
\end{Shaded}

\begin{verbatim}
##   Metric           Test
## 1    MAE      44057.497
## 2    MSE 4493345391.622
## 3     R2          0.667
\end{verbatim}

\end{document}
